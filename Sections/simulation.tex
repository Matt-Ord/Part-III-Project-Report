
\subsection{Setting up the Simulation}

Although it is theoretically possible
to evolve the system through direct
integration of the schrodinger equation

TODO-WHY NOT SCHRODINGER

Although in general \ldots hard to write
a hamiltonian for fermions \ldots
since our hamiltonian was formed from only
pairs of creation and antihalation
operators.

Once the eigenvalues and eigenvectors of the
matrix were found an arbitrary state could
be evolved in time by decomposing into by the hamiltonian

\subsection{}
Since the Hamiltonian conserved particle number
it was possible to restrict the electron basis
to states for which the total number of electrons
was constant.
\ldots electrons are fermions we either had 1 or
0 electrons
\ldots table of number of states
However in practise much of the cost
is associated with the time evolution of
the eigenstates, and as such we were limited
to around \(130\) electron states.

\subsection{Setting up the Simulation}

\subsubsection{Choosing the Initial States}
The average occupancy of the electron states
were used to select the point along the fermi
dirac distribution,
TODO- must be fermi distributed initial state
if large energy differences.
TODO- must be random initial state

and to properly match
the electron distribution within these states
at \(t=0\) the initial occupancy was therefore weighted
by the boltzmann distribution.
TODO-CODE SNIPPET
This weighting however had very little
effect when the electron states were chosen
to be very close in energy \(\Delta E \ll K_b T\)
\ldots The Temperature
at later times could then be inferred from
the distribution of electrons.

TODO- plot of electron energies at various occupation

Full code is available in the appendix TODO-


\subsection{Reducing Periodic Behaviour}
Using electrons at evenly spaced energies \ldots which lead to Rabi oscillations.

To prevent these oscillations random offsets were introduced
into the electron energy distribution, and after
averaging over several runs almost all of the
noise could be eliminated from the plot.

\subsection{}

\subsection{Adding in the energy of the hydrogen}

. Since tunnelling occurs between sates which are
degenerate in energy \ldots we don't see any tunnelling

Using the energy time uncertainty principle
\(\Delta{}E\Delta{}T \geq \frac{\hbar}{2}\), and
using a tunnelling time of \(\sim 10^{-9}s\) TODO-CITE
we are able to identify that no significant tunnelling
can occur between states separated by more than
\(\Delta{}E \sim 5\times{}10^{-26} J\). This is
much smaller than the range of energies for
which the electrons are able to undergo
transitions about the fermi surface
(\(K_b T \sim 2 \times 10^{-21}J\) at \(150K\)).


\ldots when choosing states evenly in this
range we only see degenerate tunnelling

\subsection{Multi Band Approach}
To resolve both these issues we instead simulate a system
taking the basis of electrons placed in two bands, with
the states in each band separated by at most \ldots.


When applying this approach however very
little tunnelling of the hydrogen is seen at all timescales.
Look at the energy distribution almost all
of the HCP electrons lie in the lowest energy band.
This corresponds to a system with a much lower
temperature than the initial condition.
The lack of tunnelling can therefore be explained
\ldots and if
the final states were to lie in a boltzmann  distribution
this would explain the lack of tunnelling

Running the simulation
in reverse (from HCP to FCC) the electrons
distribution \ldots population inversion.
The heat bath is therefore much to small
to accommodate the energy difference of
the two hydrogen sites.

TODO- fermi distribution
TODO- tunnelling curve
..It is therefore impossible to measure a tunnelling
rate.

TODO- The rate of tunnelling might also
be much different at such a low temperature,
and as such it is impossible to say if
any of these results would be valid
for the real system.

\subsection{Degenerate Correction Approach}
From the previous investigation it is clear
that it is not possible to incorporate the
hydrogen energy directly \ldots. In the
real system the electrons would move
between states at a much shorter time
period than that of hydrogen tunnelling.
After an electron transitions to a lower
band the fermi distribution should therefore
be recovered almost instantaneously. We
therefore `fix' the behaviour by
ignoring the fact the electron
has moved to a lower band at all,
allowing tunnelling of the hydrogen
which leaves the electron distribution
untouched. This is exactly what
we see in a system when we ignore
the hydrogen energy difference.


The Approach used to simulate the system is as follows