
Neumentical solver of lindblad/ Redfield
% equation http://qutip.org/docs/latest/guide/dynamics/dynamics-bloch-redfield.html



This is just a test

The tunnelling rate of
hydrogen

\ldots of Helium-3 spin-echo
technique~\cite{Helium_spin_echo}

Through the application of
density functional theory (DFT)
calculations the tunnelling
rates were predicted to high
accuracy at temperatures greater
than \ldots\cite{Jianding-Zhu}
At low temperatures however
the rate is dominated
by groundstate to groundstate
transitions which occur
at a much higher rate due to
their interaction with the
electron heatbath.

This report investigates
the challenges
\ldots propagating the
wavefunctions \ldots
and attempts

to propagate
..
include the electron-hydrogen
interaction \ldots previously ignored \ldots
to

The problem of propagating
a system in connection with
the environment can
be solved \ldots trace density matrix
\ldots most commonly \ldots
in the Lindblad master
equation~\cite{Manzano_2020}.

The report makes
use of such a master equation
to obtain a further
approximation of the
tunnelling rate \ldots in section \ldots.




The details of the model
are discussed in \cref{sec:the model},
with the method of
direct integration of the
resulting hamiltonian outlined
in \cref{sec:simulation}.
TODO \cref{sec:simulation results}
In \cref{sec:redfield} an
analytical approximation
of the rate was also obtained
through an application
of the redfield equation,
before TODO-Conclusion



Nickel \ldots along the closed packed plane \ldots structure~\cite{Jianding-Zhu},