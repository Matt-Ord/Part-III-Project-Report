
% Numerical solver of Lindblad/ Redfield
% equation http://qutip.org/docs/latest/guide/dynamics/dynamics-bloch-redfield.html
The aim of this report is to
investigate the extent to which
a first-principles calculation
could be used to predict the rate of
incoherent ground state to
ground state tunneling of hydrogen
on the surface of nickel.

Hydrogen diffusion has been measured
on the surface of Ni(111) %ChkTeX 36
through many different
techniques\cite{LIN199141, Ni_Diffusion_Experement},
most recently
using Helium-3 spin-echo
interferometry\cite{Helium_spin_echo}.
With the help of first-principle DFT calculations of
the H-Ni(111) potential a %ChkTeX 36
quantum version of transition state theory (ASTST)
was developed to describe these tunneling rates\cite{Jianding-Zhu}.
The results of this theory
were found to closely
match the predicted
hopping rates, particularly
when the rate was dominated
by activated tunnelling\cite{Jianding-Zhu}.
There was however a difference
in the rate at low temperatures,
where it is thought that interaction with the
electron gas leads to incoherent tunnelling.
This type of interaction
is of particular relevance in
the development of superconducting interference devices
(SQUIDs)\cite{QubitIncoherentSaito2004}
and has been successfully described
through the introduction
of effective dissipative models\cite{CALDEIRA1983374}.
To be able to assess this approach
however it is first necessary to
investigate the propagation of hydrogen directly
by introducing a model of the
complete electron-hydrogen
interaction.

To find an effective
theory describing only the system
we can trace out the
environmental degrees of
freedom to produce a quantum master
equation. Initial
attempts to find such an
equation by Redfield\cite{REDFIELD19651}
and Lamb\cite{PhysRev.134.A1429} produce
many useful results, however in
general they do not preserve the
trace of the density matrix\cite{Chru_ci_ski_2017}. This
issue was eventually solved through
the introduction of the
Lindblad equation, developed
by Lindblad\cite{Lindblad1976}
and simultaneously
by V. Gorini et al.~\cite{doi:10.1063/1.522979}.



A model for the electron-hydrogen
interaction is outlined in
\cref{sec:the model},
which is used
alongside the
Lindblad equation
to find an analytical prediction
of the tunnelling rate
in \cref{sec:redfield}.
In
\cref{sec:simulation,sec:simulation results}
the behaviour of the
system was also
investigated
through direct integration
of the
Schrödinger equation,
including both the dynamics
of the system and the environment.
The similarities and
limitations of the two
methods are discussed
in \cref{sec:improvements}
before the conclusions of
the report are
outlined in \cref{sec:conclusion}.




