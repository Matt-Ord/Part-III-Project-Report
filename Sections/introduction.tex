
Numerical solver of lindblad/ Redfield
% equation http://qutip.org/docs/latest/guide/dynamics/dynamics-bloch-redfield.html



This is just a test

The tunnelling rate of
hydrogen

\ldots of Helium-3 spin-echo
technique~\cite{Helium_spin_echo}

first application of redfield was used to
describe for nuclear magnetic resonance spectroscopy\cite{REDFIELD19651}
lamb master equation~\cite{PhysRev.134.A1429}
both of which however do not preserve the trace of the density matrix.
This method was \ldots Lindblad~\cite{Lindblad1976} and other peeps\cite{doi:10.1063/1.522979}
general overview\cite{Chru_ci_ski_2017}.

We develop a novel quantum version of the transition state theory (actual state transition
state theory), which makes use of the full bandstructure calculated from the adsorbate-surface
potential energy in a 2D or 3D space


We perform the ASTST calculations with the help of first-principle DFT calculations of
the \(H-Ni(111)\) 3D potential energy, and the resulting rates of both H and D are compared with
the experimental data. We find an excellent agreement with the experimental observation
of H diffusion, more specifically, the ASTST accurately calculates the escaping rates of H
in the activated regime, and qualitatively describes the observation of quantum tunnelling.



Through the application of

(ASTST) a precise prediction
of the tunneling rates could be
found at

Through the application of
density functional theory (DFT)
calculations the tunnelling
rates were predicted to high
accuracy at temperatures greater
than \ldots\cite{Jianding-Zhu}
At low temperatures however
the rate is dominated
by groundstate to groundstate
transitions which occur
at a much higher rate due to
their interaction with the
electron heatbath.

\cite{Manzano_2020}
Nickel \ldots along the closed packed plane \ldots structure~\cite{Jianding-Zhu},


wide range of heterogeneous catalysed processes [45], and fuel cell technology [46, 47].

\subsection{------}
The aim of this report is to
investigate the extent to which
a first-principles calculation
could be used to predict the rate of
incoherent tunneling of Hydrogen
on the surface of Nickel (Ni(111)).

Hydrogen diffusion has been measured
through a large number of
techniques\cite{LIN199141, Ni_Diffusion_Experement},
most recently
using Helium-3 spin-echo
interferometry\cite{Helium_spin_echo}. A
generalised quantum analogue of transition
state theory (ASTST) was developed to
describe these tunneling rates,
the results of which closely match the
predicted rates, particularly
in the activated regime\cite{Jianding-Zhu}.
DFT calculations used to calculate\ldots
The rate was however different at low temperatures,

\ldots difference
of which was attributed to incoherent tunnelling.


The tunneling rate of Hydrogen
is known th
With the help of first-principle DFT calculations of
the \(H-Ni(111)\) potential a
quantum version of transition state theory (ASTST)
was developed to describe these tunneling rates\cite{Jianding-Zhu}.
These calculations
found close agreement with the
experimentally observed rates, particularly
in the activated regime\cite{Jianding-Zhu}.
The rate however differed at low temperatures,
where it is thought that interaction with the
electron-gas lead to incoherent tunnelling.
To predict the incoherent behaviour
it first necessary to investigate
the propagation of hydrogen
without ignoring the electron interaction.

This problem can be
simplified by tracing over the
environmental degrees of
freedom, an idea which lead to the
development of quantum master
equations. Initial
attempts to find such an
equation by Redfield\cite{REDFIELD19651}
and Lamb\cite{PhysRev.134.A1429} produce
many useful results, however in
general they do not preserve the
trace of the density matrix. This
issue was eventually solved through
the introduction of the
Lindblad equation, developed
simultaneously by Lindblad\cite{Lindblad1976}
and others\cite{doi:10.1063/1.522979}.

A model for the Electron-Hydrogen
interaction is outlined in \cref{sec:the model},
which is investigated through
direct integration of the
Schrödinger equation in
\cref{sec:simulation}
and \cref{sec:simulation results}.
In \cref{sec:redfield} an
analytical approximation
of the rate was also obtained
through an application
of the lindblad equation.
The limitations of the two methods
are discussed in \cref{sec:improvements}
before the conclusions of
the report are outlined in \cref{sec:conclusion}.




