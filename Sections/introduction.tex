
% Numerical solver of lindblad/ Redfield
% equation http://qutip.org/docs/latest/guide/dynamics/dynamics-bloch-redfield.html

The aim of this report is to
investigate the extent to which
a first-principles calculation
could be used to predict the rate of
incoherent groundstate to
groundstate tunneling of Hydrogen
on the surface of Nickel.

Hydrogen diffusion has been measured
on the surface of Ni(111)
through a large number of
techniques\cite{LIN199141, Ni_Diffusion_Experement},
most recently
using Helium-3 spin-echo
interferometry\cite{Helium_spin_echo}.
With the help of first-principle DFT calculations of
the \(H-Ni(111)\) potential a
quantum version of transition state theory (ASTST)
was developed to describe these tunneling rates\cite{Jianding-Zhu}.
The results of this theory
were found to closely
match the predicted
hopping rates, particularly
when the rate is dominated
by activated tunnelling\cite{Jianding-Zhu}.
The rate however differed at low temperatures,
where it is thought that interaction with the
electron-gas leads to incoherent tunnelling.
To be able to predict the incoherent behaviour
it necessary to investigate
the propagation of hydrogen
without ignoring the electron-hydrogen
interaction.


This problem can be
simplified by tracing out the
environmental degrees of
freedom, an idea which lead to the
development of quantum master
equations. Initial
attempts to find such an
equation by Redfield\cite{REDFIELD19651}
and Lamb\cite{PhysRev.134.A1429} produce
many useful results, however in
general they do not preserve the
trace of the density matrix\cite{Chru_ci_ski_2017}. This
issue was eventually solved through
the introduction of the
Lindblad equation, developed
by Lindblad\cite{Lindblad1976}
and simultaneously
by V. Gorini et al.~\cite{doi:10.1063/1.522979}.



A model for the Electron-Hydrogen
interaction is outlined in
\cref{sec:the model},
which is used
alongside the
lindblad equation
to find an analytical prediction
of the tunnelling rate
in \cref{sec:redfield}.
In
\cref{sec:simulation,sec:simulation results}
the behaviour of the
system was also
investigated
through direct integration
of the
Schrödinger equation,
including both the dynamics
of the system and the environment.
The similarities and
limitations of the two
methods are discussed
in \cref{sec:lindblad assumptions,sec:improvements}
before the conclusions of
the report is
outlined in \cref{sec:conclusion}.




