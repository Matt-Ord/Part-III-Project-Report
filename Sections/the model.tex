\subsection{The Hamiltonian}
the\ldots, and contains interactions between
electrons, nickel atoms and



The \ldots was modeled using a simple
hamiltonian, assuming



\ldots system of electrons interacting with
a single hydrogen atom which would lie at
either a low or a high energy site
\begin{equation}
  \hat{H} = \hat{H}_{free} + \hat{H}_{int}
\end{equation}
where the free hamiltonian can be separated
into an electrons and hydrogen component
\begin{equation}
  \hat{H}_{free} =
  \hat{H}_{e^-} + \hat{H}_{h}\\
\end{equation}

\subsection{Electron States}
For the purpose of this model the electrons
were assumed to be free, allowing
us expand the electron
hamiltonian using the plane wave
basis
\begin{equation}
  \hat{H}_{e^-} = \sum_{k, s}
  \frac{\hbar^2 k^2}{2m_e} \hat{b}^\dagger_{k, s} \hat{b}_{k, s}\\
\end{equation}
where \(\hat{b}^\dagger_{k, s}\)
is the electron creation operator
with spin \(s\) and wavevector
\(k\) satisfying the standard
fermion anti-commutation relations
\( \{ \hat{b}_{k, s}, \hat{b}^\dagger_{k', s'} \}
= \delta_{k k'} \delta_{s s'}\).

Nickel is arranged in an FCC (cite??),
with a lattice constant of
\(3.499\pm{}0.005\times{}10^{-10}m\)~\cite{PhysRev.25.753},
which gives a Nickel density of
\(9.34\times{}10^{28}m^{-3}\).
Assuming both of the 2 valence
electrons on Nickel are completely
delocalised in
the free electron gas we
find an electron density of
\(n = 1.87\times{}10^{29} m^{-3}\).
We then used this value to calculate
the fermi-wavevector and fermi-energy of
Nickel~\cite{KittelCharles1953Itss}
\begin{align}
  \epsilon_f & = \frac{\hbar^2}{2m} {(3\pi^2n)}^{2/3} \\
  k_f^3      & = 3 \pi^2 n
\end{align}
which gives a value of
\begin{align}
  \epsilon_f & = 1.91\times{}10^{-18}J     \\
  k_f        & = 1.77\times{}10^{10}m^{-1}
\end{align}
From measurements of the optical
properties of Nickel the
fermi energy is found to
be slightly lower
at only
\(1.24\times{} 10^{-18}J\)~\cite{PhysRev.131.2469},
\ldots possibly due to the two
valence electrons not fully delocalised.

POSSIBLY IN IMPROVEMENTS SECTIONS
The true fermi surface
of Nickel is also not completely
spherical~\cite{FermiSufaceNickel},
and has a different level for different spin
cannot expand in plane waves
however \ldots we assume it is.

\subsection{Hydrogen States}
The hydrogen states produced
taking into account interaction with
the \(Ni\) lattice, with the form
of the wavefunctions taken
from previous DFT calculations. TODO-CITE

These calculations gave hydrogen
states localised at two positions
within the material, at a lower energy
`FCC' site and a higher energy
`HCP' site. We expand the hydrogen
hamiltonian in this basis
\begin{equation}
  \hat{H}_{h} =
  E_0 \hat{a}^\dagger_0 \hat{a}_0
  + E_1 \hat{a}^\dagger_1 \hat{a}_1
\end{equation}
where \(0\) denotes the FCC site and \(1\)
denotes the HCP site,
with \(\hat{a}^\dagger_i\) the hydrogen creation
operator for the site i
which satisfies the standard commutation
relations for a boson
\(\left[ \hat{a}_i, \hat{a}^\dagger_j \right]
= \delta_{ij}\).

Although the DFT calculation
also provides a theoretical calculation
of the hydrogen energies the
energy of the states used in the
model were
taken from direct spin-echo measurements.
These measurements gave a value of
\begin{equation}
  \Delta{}E_{hyd} = E_1 - E_0
  = 3.04\pm0.16\times{}10^{-21} J
  \label{eqn:hydrogen energy difference}
\end{equation}


\subsection{Electron Hydrogen Interaction}
The electron hydrogen interaction can
be described simply by introducing
the electron and hydrogen field
operators \(\hat{\psi}_e\) and
\(\hat{\psi}_h\)~\cite{nazarov_danon_2013}
\begin{equation}
  \hat{H}_{int} = \int\int{d\vec{r} d\vec{r}'
    V(\vec{r}-\vec{r}')
    \hat{\psi}^\dagger_h(\vec{r})
    \hat{\psi}^\dagger_e(\vec{r'})
    \hat{\psi}_e(\vec{r'})
    \hat{\psi}_h(\vec{r})}
\end{equation}
where \(V(\vec{r})\) is
the electron-hydrogen
interaction potential.
We first expand out the
electron operator in
the free electron basis state
\begin{align}
  \hat{\psi}_e(\vec{r}) & = \sum_{k, s}
  \braket{\vec{r}|\vec{k}}
  \hat{b}_{k, s}                                      \\
                        & = \frac{1}{L^{\frac{3}{2}}}
  \sum_{k, s} \exp{(i\vec{k}.\vec{r})}
  \hat{b}_{k, s}
\end{align}
TODO DERIVATION
\ldots giving us the
\begin{equation}
  \hat{H}_{int} = \frac{1}{L^3}
  \sum_{k,s,k',s'}
  \hat{b}^\dagger_{k',s'}\hat{b}_{k,s}
  \tilde{V}(\vec{q})\int{d\vec{r}
  \hat{\psi}_h^{\dagger}\hat{\psi}_h
  \exp(i\vec{q}.\vec{r})}
\end{equation}
We then gather the q dependant terms into
an effective potential
\begin{equation}
  \hat{H}_{int} = \sum_{k,s,k',s',i,j}
  {\tilde{V}_{eff}(\vec{q})}_{i,j}
  \hat{b}^\dagger_{k',s'}\hat{b}_{k,s}
  \hat{a}^\dagger_{i}\hat{a}_{j}
  \label{eqn:interaction hamiltonian in k}
\end{equation}

\subsection{The Electron Hydrogen Potential}

The electrons surrounding the hydrogen
atom have energies \(E_n = -\frac{13.6}{n^2} eV\) TODO-Cite.
Since the energy required to excite the electron
to the \(n=2\) energy level is much greater
than the fermi energy of \(Ni\)
(\(7.76eV\)~\cite{PhysRev.131.2469}) we can
make the approximation that the electron surrounding
the hydrogen lie close to the 1s groundstate.

The potential is then given by the equation
\begin{equation}
  V(\vec{r}) = \frac{e^2}{4 \pi \epsilon_0}(
  -\frac{1}{r}
  + \int{\frac{\abs{\phi(\vec{r}')}^2}{
      \abs{\vec{r} - \vec{r'}}} d\vec{r}'})
\end{equation}
where
\begin{equation}
  \phi(\vec{r}) = {(\pi a_0^3)}^{-1/2} e^{-\frac{r}{a_0}}
\end{equation}
is the 1s hydrogen orbital, and \(a_0\) is
the bohr radius. We then fourier
transform this expression
(see \cref{app:interaction potential calculation})
to find
\begin{eqnarray}
  V(\vec{q}) &=& \frac{e^2}{\epsilon_0 q^2}(
  \frac{\alpha^4}{{(\alpha^2 + q^2)}^2} - 1
  )
\end{eqnarray}
with \(\alpha = \frac{2}{a_0}\). If we expand
about \(q=0\) we find
\begin{eqnarray}
  V(q) &\sim&\frac{e^2}{\epsilon_0 q^2}(1 - 2{(\frac{q}{\alpha})}^2 + 3 {(\frac{q}{\alpha})}^4 - 1)\\
  {} &=& -\frac{2e^2}{\epsilon_0 \alpha^2}(1 - \frac{3}{2}{(\frac{q}{\alpha})}^2) + \mathcal{O}(q^4)
\end{eqnarray}
taking \(q = k_f = 1.175\times{}10^{10}\) TODO- ADD FERMI K CALCULATION
and \(\alpha = 3.79\times{}10^{10}\) we see the second
order correction only contributes to a variation
of around \(14.5\% \) to the overall potential.
We therefore assume that for the relevant scattering
of electrons the potential takes a constant
value \(V(q) \sim -\frac{2e^2}{\epsilon_0 \alpha^2}\).

\subsection{The Hydrogen Wavefunction}
TODO-
The form of the hydrogen wavefunctions were
known from previous \ldots DFT Calculations from Jianding's thesis \ldots


\ldots using these

The wavefunctions provided by this software however
were expressed in terms of bloch wavefunctions spread
through the whole material.
To recover a localised
wavepacket several of these states were chosen
centered about \(q=0\).
TODO-HOW
Given these wavefunctions
it was then possible to calculate the fourier
transform products directly. As there was a fixed
distance between the FCC and HCP lattice
sites however the fourier transform was seen to
oscillate, with a frequency proportional to
\(\frac{2\pi}{a}\) where \(a\) is the lattice
constant of \(Ni\). The value of this constant
is around \(3.499\pm{}0.005\times{}10^{-10}m\)~\cite{PhysRev.25.753}
corresponding to oscillations at
\(q = 1.79 \pm 0.03 \times{}10^{10}m^{-1}\).
\begin{figure}
  \centering
  \begin{subfigure}{0.45\linewidth}
    \centering
    \includegraphics[width= 0.9\linewidth]{Figures/Model/Plot of fourier transform of the wavefunction fccfcc xz plane.png}
    \subcaption{FCC-FCC Fourier Transform in xz Plane}\label{fig:diagonal hydrogen matrix element no oscillation}
  \end{subfigure}
  \hfill
  \begin{subfigure}{0.45\linewidth}
    \centering
    \includegraphics[width= 0.9\linewidth]{Figures/Model/Plot of fourier transform of the wavefunction.png}
    \subcaption{FCC-HCP Fourier Transform in xy Plane}\label{fig:cross hydrogen matrix element oscillation}
  \end{subfigure}
  \caption{Results of the hydrogen fourier transform calculations.
  FCC-FCC and HCP-HCP calculations
  (\cref{fig:diagonal hydrogen matrix element no oscillation})
  show a smooth curve, with a normalised
  value of one at \(q=0\).
  The FCC-HCP fourier transform
  (\cref{fig:cross hydrogen matrix element oscillation})
  however show oscillations with a characteristic
  wavevector \(q = 1.79 \pm 0.03 \times{}10^{10}m^{-1}\),
  corresponding to the lattice vector of \(Ni\).}
\end{figure}

In theory this should have a noticeable
effect on the matrix element, with the
fermi wavevector
and the fermi wavevector is \(1.175\times{}10^{10}\) TODO- ADD FERMI K CALCULATION
TODO- ADD PLOTS
However since the overall wavepacket decays over a region
of around \(\frac{100\pi}{a}\) we should be
able to average over the rapid oscillations of
\(V(\vec{q})\) to obtain an effective constant
potential required for the simulation.

TODO- The calculated values of the integrals!

\subsection{}
It is often beneficial to separate the interaction
hamiltonian (\cref{eqn:interaction hamiltonian in k})
into a system and environment contribution.
\begin{align}
  \hat{H}_{int} & =
  \sum_{i,j}
  \hat{a}^\dagger_{i}\hat{a}_{j}
  \sum_{k,s,k',s'}
  {\tilde{V}_{eff}(\vec{q})}_{i,j}
  \hat{b}^\dagger_{k',s'}\hat{b}_{k,s}                                                              \\
                & = \sum_{i,j} \hat{S}_{i,j} \hat{E}_{i,j}\label{eqn:split interaction hamiltonian}
\end{align}
