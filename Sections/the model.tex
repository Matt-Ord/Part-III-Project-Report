\subsection{The Hamiltonian}
the\ldots, and contains interactions between
electrons, nickel atoms and


The \ldots was modeled using a simple
hamiltonian, assuming

\ldots system of electrons interacting with
a single hydrogen atom which would lie at
either a low or a high energy site
\begin{equation}
    \hat{H} = \hat{H}_{free} + \hat{H}_{int}
\end{equation}
where the free hamiltonian is given by
\begin{eqnarray}
    \hat{H}_{free} &=& \hat{H}_{e^-} + \hat{H}_{h}\\
    \hat{H}_{e^-} &=& \sum_{k, s}
    \frac{\hbar^2 k^2}{2m_e} \hat{b}^\dagger_{k, s} \hat{b}_{k, s}\\
    \hat{H}_{h} &=&
    E_0 \hat{a}^\dagger_0 \hat{a}_0
    + E_1 \hat{a}^\dagger_1 \hat{a}_1
\end{eqnarray}
\(\hat{a}^\dagger_i\) is the hydrogen creation
operator for the site i, and satisfies the standard commutation
relations for a boson \(\left[ \hat{a}_i, \hat{a}^\dagger_j \right] = \delta_{ij}\),
and \(\hat{b}^\dagger_{k, s}\) is the electron creation operator
with spin \(s\) and wavevector \(k\) satisfying the standard
fermion anti-commutation relations
\( \{ \hat{b}_{k, s}, \hat{b}^\dagger_{k', s'} \} = \delta_{k k'} \delta_{s s'}\).

TODO- SHOW THIS
\ldots giving us the
\begin{equation}
    \hat{H}_{int} = \frac{1}{L^3}\sum_{k,s,k',s'}\hat{b}^\dagger_{k',s'}\hat{b}_{k,s}\tilde{V}(\vec{q})\int{d\vec{r}
    \hat{\psi}_h^{\dagger}\hat{\psi}_h \exp(i\vec{q}.\vec{r})}
\end{equation}
We then gather the q dependant terms into
an effective potential
\begin{equation}
    \hat{H}_{int} = \sum_{k,s,k',s',i,j}
    {\tilde{V}_{eff}(\vec{q})}_{i,j}
    \hat{b}^\dagger_{k',s'}\hat{b}_{k,s}
    \hat{a}^\dagger_{i}\hat{a}_{j}
    \label{eqn:interaction hamiltonian in k}
\end{equation}

\subsection{The Electron Hydrogen Potential}

The electrons surrounding the hydrogen
atom have energies \(E_n = -\frac{13.6}{n^2} eV\) TODO-Cite.
Since the energy required to excite the electron
to the \(n=2\) energy level is much greater
than the fermi energy of \(Ni\)
(\(7.76eV\)~\cite{PhysRev.131.2469}) we can
make the approximation that the electron surrounding
the hydrogen lie close to the 1s groundstate.

The potential is then given by the equation
\begin{equation}
    V(\vec{r}) = \frac{e^2}{4 \pi \epsilon_0}(
    -\frac{1}{r}
    + \int{\frac{\abs{\phi(\vec{r}')}^2}{
            \abs{\vec{r} - \vec{r'}}} d\vec{r}'})
\end{equation}
where
\begin{equation}
    \phi(\vec{r}) = {(\pi a_0^3)}^{-1/2} e^{-\frac{r}{a_0}}
\end{equation}
is the 1s hydrogen orbital, and \(a_0\) is
the bohr radius. We then fourier
transform this expression
(see \cref{app:interaction potential calculation})
to find
\begin{eqnarray}
    V(\vec{q}) &=& \frac{e^2}{\epsilon_0 q^2}(
    \frac{\alpha^4}{{(\alpha^2 + q^2)}^2} - 1
    )
\end{eqnarray}
with \(\alpha = \frac{2}{a_0}\). If we expand
about \(q=0\) we find
\begin{eqnarray}
    V(q) &\sim&\frac{e^2}{\epsilon_0 q^2}(1 - 2{(\frac{q}{\alpha})}^2 + 3 {(\frac{q}{\alpha})}^4 - 1)\\
    {} &=& -\frac{2e^2}{\epsilon_0 \alpha^2}(1 - \frac{3}{2}{(\frac{q}{\alpha})}^2) + \mathcal{O}(q^4)
\end{eqnarray}
taking \(q = k_f = 1.175\times{}10^{10}\) TODO- ADD FERMI K CALCULATION
and \(\alpha = 3.79\times{}10^{10}\) we see the second
order correction only contributes to a variation
of around \(14.5\% \) to the overall potential.
We therefore assume that for the relevant scattering
of electrons the potential takes a constant
value \(V(q) \sim -\frac{2e^2}{\epsilon_0 \alpha^2}\).

\subsection{The Hydrogen Wavefunction}
TODO-
The form of the hydrogen wavefunctions were
known from previous \ldots DFT Calculations from Jianding's thesis \ldots


\ldots using these

The wavefunctions provided by this software however
were expressed in terms of bloch wavefunctions spread
through the whole material. To recover a localised
wavepacket several of these states were chosen
centered about \(q=0\). Given these wavefunctions
it was then possible to calculate the fourier
transform products directly. As there was a fixed
distance between the \(FCC\) and \(HCP\) lattice
sites however the fourier transform was seen to
oscillate, with a frequency proportional to
\(\frac{2\pi}{a}\) where \(a\) is the lattice
constant of \(Ni\). The value of this constant
is around \(3.499\pm{}0.005\times{}10^{-10}m\)~\cite{PhysRev.25.753}
corresponding to oscillations at
\(q = 1.79 \pm 0.03 \times{}10^{10}m^{-1}\).

In theory this should have a noticeable
effect on the matrix element, with the
fermi wavevector
and the fermi wavevector is \(1.175\times{}10^{10}\) TODO- ADD FERMI K CALCULATION
TODO- ADD PLOTS
However since the overall wavepacket decays over a region
of around \(\frac{100\pi}{a}\) we should be
able to average over the rapid oscillations of
\(V(\vec{q})\) to obtain an effective constant
potential required for the simulation.

TODO- The calculated values of the integrals!

\section{}
It is often beneficial to separate the interaction
hamiltonian (\cref{eqn:interaction hamiltonian in k})
into a system and environment contribution.
\begin{align}
    \hat{H}_{int} & = \sum_{i,j} \hat{a}^\dagger_{i}\hat{a}_{j}
    \sum_{k,s,k',s'} {\tilde{V}_{eff}(\vec{q})}_{i,j}
    \hat{b}^\dagger_{k',s'}\hat{b}_{k,s}                                                              \\
                  & = \sum_{i,j} \hat{S}_{i,j} \hat{E}_{i,j}\label{eqn:split interaction hamiltonian}
\end{align}