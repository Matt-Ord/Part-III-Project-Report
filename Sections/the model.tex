\subsection{The Hamiltonian}
The \ldots was modeled using a simple
hamiltonian, assuming

\ldots system of electrons interacting with
a single hydrogen atom which would lie at
either a low or a high energy site
\begin{equation}
    \hat{H} = \hat{H}_{free} + \hat{H}_{int}
\end{equation}
where the free hamiltonian is given by
\begin{eqnarray}
    \hat{H}_{free} &=& \hat{H}_{e^-} + \hat{H}_{h}\\
    \hat{H}_{e^-} &=& \sum_{k, s}
    \frac{\hbar^2 k^2}{2m_e} \hat{b}^\dagger_{k, s} \hat{b}_{k, s}\\
    \hat{H}_{h} &=&
    E_0 \hat{a}^\dagger_0 \hat{a}_0
    + E_1 \hat{a}^\dagger_1 \hat{a}_1
\end{eqnarray}
\(\hat{a}^\dagger_i\) is the hydrogen creation
operator for the site i, and satisfies the standard commutation
relations for a boson \(\left[ \hat{a}_i, \hat{a}^\dagger_j \right] = \delta_{ij}\),
and \(\hat{b}^\dagger_{k, s}\) is the electron creation operator
with spin \(s\) and wavevector \(k\) satisfying the standard
fermion anti-commutation relations
\( \{ \hat{b}_{k, s}, \hat{b}^\dagger_{k', s'} \} = \delta_{k k'} \delta_{s s'}\).

TODO- SHOW THIS
\ldots giving us the
\begin{equation}
    \hat{H}_{int} = \frac{1}{L^3}\sum_{k,s,k',s'}\hat{b}^\dagger_{k',s'}\hat{b}_{k,s}\tilde{V}(\vec{q})\int{d\vec{r}
    \hat{\psi}_h^{\dagger}\hat{\psi}_h \exp(i\vec{q}.\vec{r})}
\end{equation}
We then gather the q dependant terms into
an effective potential
\begin{equation}
    \hat{H}_{int} = \sum_{k,s,k',s',i,j}
    {\tilde{V}_{eff}(\vec{q})}_{i,j}
    \hat{b}^\dagger_{k',s'}\hat{b}_{k,s}
    \hat{a}^\dagger_{i}\hat{a}_{j}
    \label{eqn:interaction hamiltonian in k}
\end{equation}

\subsection{The Electron Hydrogen Potential}

The electron surrounding the hydrogen was assumed
to lie in the 1s orbital
TODO-SHOW ENERGY TO DISSOCIATE IS TOO LARGE

The potential is then given by the equation
\begin{equation}
    V(\vec{r}) = \frac{e^2}{4 \pi \epsilon_0}(
    -\frac{1}{r}
    + \int{\frac{\abs{\phi(\vec{r}')}^2}{
            \abs{\vec{r} - \vec{r'}}} d\vec{r}'})
\end{equation}
where
\begin{equation}
    \phi(\vec{r}) = {(\pi a_0^3)}^{-1/2} e^{-\frac{r}{a_0}}
\end{equation}
is the 1s hydrogen orbital, and \(a_0\) is
the bohr radius. We then fourier
transform this expression
(see \cref{app:interaction potential calculation})
to find
\begin{eqnarray}
    V(\vec{q}) &=& \frac{e^2}{\epsilon_0 q^2}(
    \frac{\alpha^4}{{(\alpha^2 + q^2)}^2} - 1
    )
\end{eqnarray}
with \(\alpha = \frac{2}{a_0}\). If we assume scattering
occurs between states separated by at most
twice the fermi energy \(q \leq 2k_f\) we find
TODO- JUSTIFY Q=0

\subsection{The Hydrogen Wavefunction}
TODO-
The form of the hydrogen wavefunctions were
known from previous

\ldots using these

To produce a localised wavepacket several of these
sates are chosen around \(q=0\)

It was then possible to calculate the fourier
transform directly

The calculation gave

Although the fourier transform varied was not constant
the rate
TODO-JUSTIFY