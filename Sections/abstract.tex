A first principles
approach was used to
model the incoherent groundstate to
groundstate tunneling of
hydrogen on a Ni(111) surface.

Through the application of the
Lindblad equation the model
was used to produce a closed form
of the hydrogen density
matrix evolution equation
\begin{equation}
    \bra{m}\dot{\hat{\rho}}(t) \ket{m}  =
    2 \sum_{n\neq m}
    [  \Gamma_{m,n;m, n}(\omega_{m,n})\rho_{n, n}
        - \Gamma_{n,m;n, m}(\omega_{n,m})\rho_{m, m}]
\end{equation}
where
\begin{equation}
    \Gamma_{i,j, k,l}(\omega_{k,l})   =
    \exp{(\frac{\beta \omega_{k,l}}{2})}
    \mathcal{C}_{i,j} \mathcal{C}_{k,l}
    \sqrt{\pi} \frac{32 k_f^2 \epsilon_0^2 \hbar^3}{\beta e^4 m_e^2}
\end{equation}
\(\omega_{i,j} = E_i - E_j\),
\(E_i\) is the energy of the hydrogen at
site \(i\) and \(k_f\) is the fermi wavevector
of Nickel. Previous DFT calculations
were used to find the hydrogen
overlap fraction
\(\mathcal{C}_{i,j}\) where
\(i\), \(j\) denotes either
a low energy FCC or a
high energy HCP site\cite{Jianding-Zhu}.
Analysis of the combined FCC and HCP
occupation gave an expression
for the tunneling rate \(R\)
\begin{equation}
    R   =
    12\cosh{(\frac{\beta \omega_{k,l}}{2})}
    \mathcal{C}_{1,0}\mathcal{C}_{0,1}
    \sqrt{\pi} \frac{32 k_f^2 \epsilon_0^2 \hbar^3}{\beta e^4 m_e^2}
\end{equation}
where the FCC-HCP
overlap of Nickel
\(\mathcal{C}_{1,0}\)
was found to be \(\sim 4.4\times{}10^{-3}\).
Calculations at
\(150K\) gave a rate of
\(1.8\times 10^9s^{-1}\),
which was close to the
measured value of \(3.3\times 10^9s^{-1}\)\cite{Jianding-Zhu}.
Further measurements of the
rate from initial site
occupation \(2.5\times 10^{9}s^{-1}\), HCP occupation
\(2.3\times 10^{9}s^{-1}\)
next FCC occupation
\(1.5\times 10^{9}s^{-1}\)
and mean-squared distance
\(2.0\times{} 10^9s^{-1}\)
were also shown to be constant
with experiment.

The full electron-hydrogen
dynamics were then investigated
through direct integration
of the schrodinger equation.
Data extracted from the simulation
predicted a tunneling rate of
\(4.3\pm 0.1\times10^{9}s^{-1}\)
at 150K,
which was again
consistent with
experiment. Comparison
between the behaviour
of simulation
and Lindblad analysis
found many similarities
in the
form of the predicted
hydrogen dynamics.