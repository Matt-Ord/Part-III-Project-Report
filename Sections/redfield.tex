
\subsection{}
TODO-BACKGROUND INTRODUCTION, ASSUMPTIONS

\subsection{}
starting from the redfield equation
\begin{equation}
  \dot{\hat{\rho}}_S(t) =
  - \int_0^{\infty} ds
  Tr_{E}[\hat{H}_{int}(t),
      [\hat{H}_{int}(s-t),
          \hat{\rho}_S(t) \otimes \hat{\rho}_E(t)]]
\end{equation}
where \(\hat{\rho}_S(t)\) is the
density matrix of the hydrogen system
and \(\hat{\rho}_E(t)\) is the density
matrix of the electron system.

TODO-system and environment

\subsection{Calculating \(\Gamma \)}
Using expression \ldots gamma could then
be calculated.

\subsection{Plotting }
Under the rotating wave

the

\subsection{}
If we relax the rotating wave approximation
we find the diagonal terms evolve as
\begin{equation}
  \dot{P(t)}_{\mu, \mu} = \begin{aligned}[t]
      & \Gamma_{\mu, \nu; \mu, \nu}(\omega_l)
    (P_{\nu, \nu}(t) + P_{\nu, \nu}(t))       \\
    + & \exp{(i(\omega_{\nu \mu})t)}
    \Gamma_{\mu, \mu; \mu, \nu}(\omega_l)
    (P_{\mu, \nu}(t) + P_{\nu, \mu}(t))       \\
    - & \exp{(i(\omega_{\mu \nu})t)}
    \Gamma_{\nu, \nu; \nu, \mu}(\omega_l)
    (P_{\nu, \mu}(t) + P_{\mu, \nu}(t))       \\
    - & \Gamma_{\nu, \mu; \nu, \mu}(\omega_l)
    (P_{\mu, \mu}(t) + P_{\mu, \mu}(t))
  \end{aligned}
\end{equation}
and the cross terms evolve as
\begin{equation}
  \dot{P(t)}_{\mu, \nu}= \begin{aligned}[t]
    [ & P_{\nu, \nu}(t)(
      \exp{(i(\omega_{\mu \nu})t)}
    (- \Gamma_{\nu, \nu; \nu, \mu}(\omega_l))                                                   \\
    + & \exp{(i(\omega_{\nu \mu})t)}
    \Gamma_{\nu, \nu; \mu, \nu}(\omega_l) )                                                     \\
    + & P_{\mu, \nu}(t)(
    - \Gamma_{\mu, \nu; \mu, \nu}(\omega_l) - \Gamma_{\nu, \mu; \nu, \mu}(\omega_l))            \\
    + & P_{\nu, \mu}(t)(
    \exp{(2i(\omega_{\nu \mu})t)}\Gamma_{\nu, \mu; \mu, \nu}(\omega_l)                          \\
      & +                   \exp{(2i(\omega_{\mu \nu})t)}\Gamma_{\mu, \nu; \nu, \mu}(\omega_l)) \\
    + & P_{\mu, \mu}(t) (
    \exp{(i(\omega_{\mu \nu})t)}\Gamma_{\mu, \mu; \nu, \mu}(\omega_l)                           \\
      & + \exp{(i(\omega_{\nu \mu})t)}[
        - \Gamma_{\mu, \mu; \mu, \nu}(\omega_l)])]
  \end{aligned}
\end{equation}
Although it is not possible to solve
this analytically, when plotting the
equation numerically we see
exactly the same same long-timescale
behaviour as that predicted by the
rotating wave approximation. Although the density
matrix could be seen to oscillate at a timescale of
\(10^{-13}s\) this had no effect on the macroscopic
behaviour of the density matrix.

TODO-plots and proof??
\subsection{}

One issue with the \ldots approach is that it failed to
incorporate behaviour key to incoherent tunnelling.
In the simulation the tunnelling process was dominated by
transitions between two states with the same energy,
rather than two states with the same electron configuration.
To have any hope of recovering this behaviour we therefore need
to modify the approach to take this into account.







