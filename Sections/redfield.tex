
\subsection{General Equation of Motion}
The state of the electron hydrogen
system at a given time can be completely
characterised by its density
matrix.
Working in the interaction
picture, a general density matrix
\(\hat{\rho}(t)\) time evolves according
to the von Neumann equation~\cite{TP2_Notes}.
\begin{equation}
    \frac{d\hat{\rho}_t(t)}{dt} =
    -i [\hat{H}_{int}(t), \hat{\rho}_t(t)]
    \label{eqn:density equation of motion}
\end{equation}
which can be integrated to give
\begin{equation}
    \hat{\rho}_t(t) =
    \hat{\rho}_t(0)
    - i \int_0^t ds
        [\hat{H}_{int}(s), \hat{\rho}_t(s)]
    \label{eqn:integrated density equation of motion}
\end{equation}
We can expand this equation of motion
to second order in the interaction
by substituting \cref{eqn:integrated density equation of motion}
into \cref{eqn:density equation of motion}
twice to give
\begin{equation}
    \frac{d\hat{\rho}_t(t)}{dt} =
    -i [\hat{H}_{int}(t), \hat{\rho}_t(0)]
    - \int_0^t ds
        [\hat{H}_{int}(s),
            [\hat{H}_{int}(s), \hat{\rho}_t(t)]]
    +\mathcal{O}({\hat{H}_{int}}^3)
\end{equation}
It is possible to reduce this to
an equation of motion
describing just the system by taking
a trace over the environment~\cite{Manzano_2020}
\begin{equation}
    \hat{\rho}(t) =
    -i Tr_e[\hat{H}_{int}(t), \hat{\rho}_t(0)]
    - \int_0^t ds
    Tr_e[\hat{H}_{int}(s),
    [\hat{H}_{int}(s), \hat{\rho}_t(t)]]
    \label{eqn:density motion before redfield approximation}
\end{equation}
where \(\hat{\rho}(t) = Tr_e[\hat{\rho}_t(t)]\)
is the density operator of the system.
Using a clever re-definition of the interaction
Hamiltonian~\cite{Manzano_2020} it
is possible to show that the first
term gives no contribution to the
overall dynamics of the system.

\subsection{The Redfield Assumption}\label{sec:the redfield assumption}
To arrive at the Redfield equation
we first make the assumption that the
system and surrounding density
matrix is completely
decoupled~\cite{theory_open_quantum_systems},
allowing us to write
\begin{equation}
    \hat{\rho}_t(t) = \hat{\rho}(t) \otimes \hat{\rho}_E(t)
\end{equation}
where \(\hat{\rho}_E(t)\), the
density matrix of the environment,
is taken as a purely statistical ensemble.
Under the Markov approximation
we can extend the upper limit
of \cref{eqn:density motion before redfield approximation}
to \(\infty \), arriving at the Redfield
equation
\begin{equation}
    \dot{\hat{\rho}}(t) =
    - \int_0^{\infty} ds
    Tr_{E}[\hat{H}_{int}(t),
            [\hat{H}_{int}(s-t),
                    \hat{\rho}(t) \otimes \hat{\rho}_E(t)]]
\end{equation}
Separating out the interaction hamiltonian
into system and surroundings according
to~\cref{eqn:split interaction hamiltonian}
\begin{align}
    \hat{H}_{int} & = \sum_{i,j} \hat{S}_{i,j} \hat{E}_{i,j}
\end{align}
we can simplify the form of this equation to give~\cite{Manzano_2020}
\begin{equation}
    \dot{\hat{\rho{}}}(t) = \begin{aligned}[t]
        \sum_{i,j,k, l} &
        \exp{(-i(\omega_{i,j}-\omega_{k,l})t)}
        \Gamma_{i,j;k, l}(\omega_{k,l})
        [S_{k, l}\hat{\rho}(t),
        S^\dagger_{i,j}]  \\
        +               &
        \exp{(i(\omega_{i,j}-\omega_{k,l}))}
        \Gamma^*_{k, l; i,j}(\omega_{i,j})
        [S_{k, l},
            \hat{\rho}(t) S^\dagger_{i,j}]
    \end{aligned} \label{eqn:redfield equation gamma form}
\end{equation}
where \(\Gamma \) is given by
\begin{equation}
    \Gamma_{i,j, k,l}(\omega) =
    \int_0^\infty{}{
    ds \exp{(i\omega{}s)}
    Tr_{E}[E^\dagger_{i,j}(t)E_{k,l}(t-s)\rho_E(0)]
    }\label{eqn:gamma definition}
\end{equation}

\subsection{The Lindblad Equation}
To obtain the Lindblad equation
we need to apply the rotating
wave approximation to
\cref{eqn:redfield equation gamma form}.
Before applying this
we first expand out the commutators
(reference)
\begin{equation}
    \bra{m}[S_{k, l}\hat{\rho}(t),
        S^\dagger_{i, j}] \ket{n} =
    \sum_{\alpha, \beta} \rho_{\alpha, \beta} [
        \delta_{m, k}\delta_{l, \alpha}
        \delta_{\beta, j}\delta_{i, n}
        -\delta_{m, j}\delta_{i, k}
        \delta_{l, \alpha}\delta_{\beta, n}]
\end{equation}

to

\subsection{Calculating \(\Gamma \)}
Using \cref{eqn:gamma definition}
we can calculate the value of \(\Gamma \).
The
density matrix of a purely
statistical ensemble is given
by~\cite{sakurai_napolitano_2020}
\begin{equation}
    \rho_E(0) = \sum_{\{N(k)\}}
    P(\{N(k)\})
    \ket{N(k)} \bra{N(k)}
\end{equation}
and
\begin{equation}
    \hat{E}_{i,j} = \sum_{k,s,k',s'}
    {\tilde{V}_{eff}(\vec{q})}_{i,j}
    \hat{b}^\dagger_{k',s'}\hat{b}_{k,s}
\end{equation}
where we assume the potential
is independent of \(q\). We can
take the trace over the
environment (todo reference)
\begin{equation}
    Tr_E[\dots]  = \begin{aligned}[t]
        \sum_{\substack{\{N(k)\}                             \\
        k_1,s^1,k_2,s^2                                      \\
                k_3,s^3,k_4,s^4 }}
         & P(\{N(k)\}) V_{i,j} V_{k,l}                       \\
         & \exp{(i(E_1 - E_2) t)} \exp{(i(E_3 - E_4) (t-s))} \\
         & \bra{N(k)}
        \hat{b}_{k_1,s^1}^\dagger{} \hat{b}_{k_2,s^2}
        \hat{b}_{k_3,s^3}^\dagger{} \hat{b}_{k_4,s^4}
        \ket{N(k)}
    \end{aligned}
\end{equation}
where \( \{N(k)\} \) is the set of
all possible occupations, and
the boltzmann
probability associated with a
given state is
\(P(\{N(k)\}) = \exp{(-\beta{}(E-\mu N))}\).
The trace is only non zero in two
cases
\begin{itemize}
    \item \(k_1=k_2, s^1=s^2\),
          \(k_3=k_4, s^3=s^4\)
    \item \(k_1=k_4, s^1=s^4\),
          \(k_3=k_2, s^3=s^2\) but
          \(k_1\neq{}k_2, s^1\neq{}s^2\)
\end{itemize}
and we can therefore obtain the
simplified form of the trace (reference)
\begin{equation}
    Tr_E[\dots] = \begin{aligned}[t]
        \sum_{k_1,s^1,k_3,s^3 }
         & V_{i,j} V_{k,l} [ \\
         & N_1 N_3
                + N_1 (1 - N_3) \exp{(-i(E_3 - E_1)s)}]
    \end{aligned}
\end{equation}
Integrating over \(s\) we obtain
an additional constant on the
wavevectors, and after
converting the sum into
an integral and re-adsorbing
the power of \(L^3\) into the
definition of (reference)
\begin{align}
    \Gamma_{i,j, k,l}(\omega) & =\begin{aligned}[t]
        \sum_{s^1,s^3} \int &
        \frac{d^3\vec{k}_1}{{(2\pi)}^3}
        \frac{d^3\vec{k}_3}{{(2\pi)}^3}
        V_{i,j} V_{k,l} [
        N_1 N_3 \delta_{w, 0} \frac{m}{\sqrt{k_3^2}} \\
                            & + N_1 (1 - N_3)
                \frac{m}{\sqrt{k_1^2 - 2m\omega}}
                \delta({k_3 \pm \sqrt{k_1^2 + 2m\omega}}) ]
    \end{aligned}
\end{align}
The first term is divergent (reference)
but we find no terms in the \ldots
with \(\omega = 0\).
We evaluate the second term
by expanding about the fermi
wavevector
\begin{equation}
    \Gamma_{i,j, k,l}(\omega_{k,l}) =\begin{aligned}[t]
        \sum_{s^1,s^3} \exp{(\frac{\beta \omega_{k,l}}{2})} \frac{m k_f^2 }{{(2\pi)}^4}
        V_{i,j} V_{k,l} \sqrt{\pi} \frac{2m}{\beta \hbar^2}
    \end{aligned}
\end{equation}
substituting in the expression
for v (reference)
we arrive at the final
expression for \(\Gamma \)
\begin{equation}
    \sum_{s^1,s^3} \exp{(\frac{\beta \omega_{k,l}}{2})}
    \mathcal{C}_{i,j} \mathcal{C}_{k,l}
    \sqrt{\pi} \frac{8 k_f^2 \epsilon_0^2 \hbar^3}{\beta e^4 m^2}
\end{equation}

\subsection{}
we arrive at the final form of
the Lindblad equation
for our system
\begin{equation}
    \bra{m}\dot{\hat{\rho{}(t)}} \ket{m} = \begin{aligned}[t]
        [ & 2\Gamma_{m,\neq m;m, \neq m}(\omega_{m,\neq m})\rho_{\neq m, \neq m} \\
        - & 2\Gamma_{\neq m,m;\neq m, m}(\omega_{\neq m,m})\rho_{m, m}]
    \end{aligned}
\end{equation}
where
\begin{equation}
    \Gamma_{i,j, k,l}(\omega_{k,l})   =
    \exp{(\frac{\beta \omega_{k,l}}{2})}
    \mathcal{C}_{i,j} \mathcal{C}_{k,l}
    \sqrt{\pi} \frac{32 k_f^2 \epsilon_0^2 \hbar^3}{\beta e^4 m^2}
\end{equation}

\subsection{Analytic Solution to the Rotating Wave Approximation}
Since the form of the rotating
wave approximation is a simple
rate equation with two variables
it is possible to solve it analytically
\cref{app:combined tunnelling rates}.
From the expression above we find the
forward and backward tunnelling rate as
\begin{align}
    \gamma_0 & = 2\Gamma_{1,0;0, 1}(\omega_{1,0})       \\
             & = A \exp{(\frac{\beta \omega_{0,1}}{2})}
    \mathcal{C}_{1,0} \mathcal{C}_{0,1}                 \\
    \gamma_1 & = 2\Gamma_{0,1;1, 0}(\omega_{0,1})       \\
             & = A \exp{(\frac{\beta \omega_{1,0}}{2})} \\
\end{align}
where
\(A =
\mathcal{C}_{1,0} \mathcal{C}_{0,1}
\sqrt{\pi}
\frac{64 k_f^2 \epsilon_0^2 \hbar^3}{\beta e^4 m^2}\).
This gives a combined rate of
\begin{equation}
    \gamma_0 + \gamma_1 = 2A\cosh{(\frac{\beta (E_1 - E_0)}{2})}
    \label{eqn:theoretical rate lindblad equation}
\end{equation}
for an energy difference of
\(3.04\pm0.16\times{}10^{-21} J\)
(\cref{eqn:hydrogen energy difference})
we find a tunnelling rate of
\(6.1\times{}10^{8}s^{-1}\),
corresponding to a
tunnelling time of
\(1.6\times{}10^{-9}s\).


\begin{figure}
    \centering
    \includegraphics[width=.5\linewidth]{Figures/Redfield/Plot of lindblad solution.png}
    \caption{Plot of the Lindblad solution with a characteristic decay
    rate of \(6.1\times{}10^{8}s^{-1}\) TODO:Is this a typo???
    }\label{fig:two site lindblad soluton}
\end{figure}


\subsection{}
In reality there
are 3 HCP sites neighbouring
each FCC Hydrogen, all of which
are connected to 2 other HCP sites.
The tunnelling
rate should therefore be at least
\(3\) times the single
neighbour rate TODO CITE MODEL CHAPTER.
To investigate this behaviour we extend
the simulation to contain a large
grid of sites with periodic boundary conditions.
From this we find that the combined FCC occupation
falls at a rate of
\(1.8\times{}10^{9}s^{-1}\), exactly
three times the single neighbour rate (\cref{fig:multi site lindblad}).
\begin{figure}[htbp]
    \centering
    \begin{subfigure}{0.45\linewidth}
        \centering
        \includegraphics[width =0.9 \linewidth]{Figures/Redfield/Plot of lindblad solution many sites.png}
        \caption{Individual occupation probability
        }\label{sub@fig:multi site lindblad}
    \end{subfigure}
    \hfill
    \begin{subfigure}{0.45\linewidth}
        \centering
        \includegraphics[width = 0.9\linewidth]{Figures/Redfield/Plot of redfield solution long time.png}
        \caption{Combined occupation probability
        }\label{sub@fig:multi site combined lindblad}
    \end{subfigure}
    \caption{Plot of the individual and combined
    occupation probabilities against time. The combined
    probability
    (\cref{sub@fig:multi site combined lindblad})
    follows exactly the same curve as in
    \cref{fig:two site lindblad soluton}
    with a rate of \(1.8\times{}10^{9}s^{-1}\).
    The plot of the individual occupation
    probabilities
    (\cref{sub@fig:multi site lindblad})
    shows that it takes
    a much longer time for the
    probability of occupation of the
    initial site to reach
    an equilibrium with the surroundings.}\label{fig:multi site lindblad}
\end{figure}
The time taken for the probability
at the initial site to reach
equilibrium is however significantly
longer, and there are several
features of the plot
which could be identified as a
tunneling time.
\begin{itemize}
    \item
\end{itemize}

And their corresponding
tunneling times are




\subsection{Extracting a Tunnelling Rate}
One issue is how best to characterise the
true tunnelling rate of this system\ldots
however the decoherent process is completely
ignored in this model

\subsection{Temperature Dependance}
\ldots we can now compare \ldots
to the temperature dependant
rates seen in experiment.

\subsection{Distance Traveled}
As well as the correct
temperature dependance
we also expect the squared
distance travelled to
grow linearly.
\begin{figure}
    \centering
    \includegraphics[width=0.5\linewidth]{Figures/Redfield/Plot of lindblad solution squared distance.png}
    \caption{Plot of the squared distance
    of the hydrogen atom against time
    showing a linear trend as expected
    for a random walk. The time taken
    for the rms distance to equal \(0.5\)
    is found to be
    \(5.25\times{}10^{-10}s\),
    which corresponds to an implied rate
    of \(1.9 \times 10^{9}s^{-1}\).
    }
\end{figure}
From this data we are able to
extract another measure
of the tunneling rate. The
time taken for the rms
distance to reach \(0.5\)
is found to be \(5.25\times{}10^{-10}s\)
giving an overall rate of
\(1.9 \times 10^{9}s^{-1}\).


\subsection{Assessing the Rotating Wave approximation}
If we relax the rotating wave approximation
we arrive at the expression given in \cref{sec:redfield equation full solution}.
\begin{align}
    \bra{m}\dot{\hat{\rho{}}}(t) \ket{n} & = \begin{aligned}[t]
        \sum_{i,j} &
        \exp{(-i\Delta{}E_{n,j;m,i} t)}
        \Gamma_{n,j;m, i}(\omega_{m,i})
        \rho_{i,j}   \\
                   &
        -\exp{(-i\Delta{}E_{i,m;i,j} t)}
        \Gamma_{i,m;i, j}(\omega_{i,j})
        \rho_{j, n}  \\
                   &
        +\exp{(i\Delta{}E_{n,j;m,i} t)}
        \Gamma_{n,j; m, i}(\omega_{n,j})
        \rho_{i, j}  \\
                   &
        -\exp{(i\Delta{}E_{i,j;i,n} t)}
        \Gamma_{i,j; i, n}(\omega_{i,j})
        \rho_{m, j}
    \end{aligned}
\end{align}
This equation produces extra oscillations
on top of the Lindblad result, with
a characteristic timescales of
\(\frac{2\pi}{\omega_{1,0}} = 2.13\times{}10^{-13}s\). Plotting
the full solution (\cref{fig:redfield full solution})
however we see exactly the same behaviour as that
predicted by the lindblad result.

\begin{figure}[htbp]
    \centering
    \begin{subfigure}{0.45\linewidth}
        \centering
        \includegraphics[width =0.9 \linewidth]{Figures/Redfield/Plot of redfield solution short time.png}
        \caption{Complete solution for small times
        }\label{fig:redfield full solution short timescales}
    \end{subfigure}
    \hfill
    \begin{subfigure}{0.45\linewidth}
        \centering
        \includegraphics[width = 0.9\linewidth]{Figures/Redfield/Plot of redfield solution long time.png}
        \caption{Complete solution for long times
        }\label{fig:redfield full solution long timescales}
    \end{subfigure}
    \caption{Plot of the full solution of the Redfield
    equation. On short timescales
    (\cref{fig:redfield full solution short timescales})
    the solution is seen to
    oscillate with a characteristic
    frequency of \(2.1\times{}10^{-13}\)s however
    at long timescales (\cref{fig:redfield full solution long timescales})
    the solution decays at the same rate as the
    Lindblad equation.}\label{fig:redfield full solution}
\end{figure}

In theory we should also be able
to solve the redfield equation
for multiple hydrogen sites, however
the additional computational complexity
rules this out. It is however possible to
approximate the behaviour
seen in the many site model
by placing a sink at
the HCP site
(\cref{fig:redfield full solution with sink}).
In this case
we again see good
agreement with the
lindblad equation.
\begin{figure}[htbp]
    \centering
    \includegraphics[width =0.45 \linewidth]{Figures/Redfield/Plot of redfield solution long time sink.png}
    \caption{Plot of the full solution of the Redfield
        equation with a sink on the HCP site.
        The hydrogen behaves exactly the same
        as in the previous lindblad analysis.}
\end{figure}



\subsection{Further Generalisations}
One issue with this approach is that it
completely ignores correlations
between the system and the surroundings.
This contradicts
one of the key features seen in the simulation;
the tunnelling process is dominated by
transitions between two states with the same energy,
rather than two states with the same electron configuration.
It is not possible to `trace out' the environment
for an arbitrary coupling, however if
we limit ourselves to
\begin{equation}
    \hat{\rho}_t = \sum_{m,n} \hat{\rho}_{m,n} \otimes {(\hat{\rho}_E)}_{m,n} \
\end{equation}
we can follow the same procedure as in
\cref{sec:the redfield assumption}
to arrive at the modified redfield equation
\begin{equation}
    \bra{m}\dot{\hat{\rho}}(t)\ket{n} = \begin{aligned}[t]
        \sum_{i,j,k, l} &
        \exp{(-i(\omega_{i,j}-\omega_{k,l})t)}
        \Gamma^{m,n}_{i,j;k, l}(\omega_{k,l})
        [S_{k, l}{\hat{\rho}(t)}_{m,n},
        S^\dagger_{i,j}]  \\
        +               &
        \exp{(i(\omega_{i,j}-\omega_{k,l}))}
        {\Gamma^*}^{m,n}_{k, l; i,j}(\omega_{i,j})
        [S_{k, l},
                {\hat{\rho}(t)}_{m,n} S^\dagger_{i,j}]
    \end{aligned}
\end{equation}
where
\begin{equation}
    \Gamma^{m,n}_{i,j, k,l}(\omega) =
    \int_0^\infty{}{
    ds \exp{(i\omega{}s)}
    Tr_{E}[E^\dagger_{i,j}(t)E_{k,l}(t-s)
    {(\hat{\rho}_E)}_{m,n}]
    }
\end{equation}
The problem is then how best to express
both the statistical and quantum uncertainty
in the form of a density matrix.

\subsection{Energy Conservation}
TODO- Reduced Temperature density matrix

TODO- How do we produced thematised and localised states??
we can do one but not both??
TODO- we want states to lie close to the fermi level after
perturbation


