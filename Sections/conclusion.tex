A simple model of
an electron-hydrogen
system on Ni
was developed
to explain the
incoherent tunneling
of hydrogen on a
Ni(111) surface. %ChkTeX 36

Through an application
of the Lindblad equation
a simple, closed form
expression for
the hydrogen density matrix
evolution was found,
which was used to
find an expression
for the rate at which
the combined
FCC occupation reached
equilibrium
\begin{equation}
    R   =
    12\cosh{(\frac{\beta \omega_{k,l}}{2})}
    \mathcal{C}_{1,0}\mathcal{C}_{0,1}
    \sqrt{\pi} \frac{32 k_f^2 \epsilon_0^2 \hbar^3}{\beta e^4 m_e^2}
\end{equation}
where \(\omega_{i,j}=E_i - E_j\),
\(E_i\) is the energy at site \(i\),
\(\mathcal{C}_{i,j}\) is the hydrogen
overlap fraction.
Rates extracted from
the combined FCC
occupation (\(1.8\times{}10^{-9}s^{-1}\))
next HCP occupation
(\(2.3\times{}10^{-9}s^{-1}\))
next FCC occupation
(\(1.5\times{}10^9s^{-1}\))
and initial FCC occupation
(\(2.5\times{}10^9s^{-1}\))
were all close to the
experimental rate
of \(3.3 \times{}10^{9}s^{-1}\)
at 150K\cite{Jianding-Zhu}.
The mean squared distance
was also shown to increase
linearly with time, as expected
from a decoherent stochastic
process. From this
a tunneling rate of
\(2 \times{}10^{9}s^{-1}\)
was extracted,
which was also constant with
the data at 150K.

The behaviour of the
full electron-hydrogen
model was then
investigated through
direct integration of
the Schrödinger equation.
It was found that a
simulation
including the energy of
hydrogen was not possible,
and instead a model
of degenerate hydrogen
was used to measure the
tunneling rates.
This model predicted
a tunneling rate of
\(3.6\pm 0.1\times{}10^{9}s^{-1}\)
at \(150K\)
which gave a good
prediction of the
experimental value
of
\(3.3 \times{}10^{9}s^{-1}\)\cite{Jianding-Zhu}.
A corrected tunneling rate
made with comparison
to the Lindblad equation
was
however slightly
higher at
\(4.3\pm 0.1\times{}10^{9}s^{-1}\).
The corrected rate was shown to have
a temperature dependance
consistent with that seen in
experiment
within the incoherent regime.



Results of the simulation
were then applied
to assess some of the
approximations
made in the lindblad equation,
and it was argued that
the off-diagonal
elements of the electron
density matrix
had no effect on the tunneling
rate. The simulation
also provided evidence
that the second order approximation
used in the lindblad analysis
was correct,
as both methods found a
rate proportional to \(N(1-N')\).

To test
this model further
it should also be possible
to apply the same
techniques
to predict the
groundstate to
groundstate tunneling
of deuterium, however
there is currently
discussion
surrounding the
accuracy of the
experimental data.
The results could
also be used to
investigate the
H-Ru(111) and D-Ru(111) %ChkTeX 36
however the fermi
surface of Ruthenium
is much less spherical
and hence
the free electron
approximation may
no longer hold.

To provide more precise
predictions of the
tunneling rate
it may also be necessary
to investigate the
extent to which q-dependance
of both the interaction
and the hydrogen overlap
effects the model. It is thought
that this could cause a
significant reduction in
the calculated tunneling
rate.





