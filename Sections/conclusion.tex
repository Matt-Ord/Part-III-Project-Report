A simple model of
an electron-hydrogen
system on Ni
was developed
to explain the
incoherent tunneling
of hydrogen on a
Ni(111) surface.

Through an application
of the Lindblad equation
a simple, closed form
expression for
the hydrogen density matrix
evolution was found,
which was used to
find an expression
for the rate at which
the combined
FCC occupation reached
equilibrium
\begin{equation}
    R   =
    12\cosh{(\frac{\beta \omega_{k,l}}{2})}
    \mathcal{C}_{1,0}\mathcal{C}_{0,1}
    \sqrt{\pi} \frac{32 k_f^2 \epsilon_0^2 \hbar^3}{\beta e^4 m_e^2}
\end{equation}
where \(\omega_{i,j}=E_i - E_j\),
\(E_i\) is the energy at site \(i\),
\(\mathcal{C}_{i,j}\) is the hydrogen
overlap fraction.
Rates extracted from
the combined FCC
occupation (\(1.8\times{}10^{-9}s^{-1}\))
next HCP occupation
(\(2.3\times{}10^{-9}s^{-1}\))
next FCC occupation
(\(1.5\times{}10^9s^{-1}\))
and initial FCC occupation
(\(2.5\times{}10^9s^{-1}\))
were all close to the
experimental rate
of \(3.3 \times{}10^{9}s^{-1}\)
at 150K\cite{Jianding-Zhu}.
The mean squared distance
was also shown to increase
linearly with time, as expected
from a decoherent stochastic
process. From this
a tunneling rate of
\(2 \times{}10^{9}s^{-1}\)
was extracted,
which was also constant with
the data at 150K.
To provide a more precise
prediction of the
tunneling rate however
it will be necessary
to include the q-dependance
of both the interaction
and the hydrogen overlap
integral, as it is thought
that this could cause a
significant reduction in
the calculated tunneling
rate.

The behaviour of the
full electron-hydrogen
model was then
investigated through
direct integration of
the schrodinger equation.
It was found that a
simulation
including the energy of
hydrogen was not possible,
and instead a model
of degenerate hydrogen
was used to measure the
tunneling rates.
This model predicted
a tunneling rate of
\(3.6\pm 0.1\times{}10^{9}s^{-1}\)
at \(150K\)
which gave a good
prediction of the
experimental value
of
\(3.3 \times{}10^{9}s^{-1}\)\cite{Jianding-Zhu}.
A corrected tunneling rate
made with comparison
to the Lindblad equation
was
however slightly
higher at
\(4.3\pm 0.1\times{}10^{9}s^{-1}\).
The corrected rate was shown to have
a temperature dependance
consistent with that seen in
experiment
within the incoherent regime.



Results of the simulation
were then applied
to assess some of the
approximations
made in the lindblad equation.

Both methods found a rate
proportional to \(N(1-N')\),
consistent with the second
order approximation made
in the lindblad analysis.
We also find that the electron
density matrix \ldots

Comparison between
the lindblad equation
and the simulation
direct







