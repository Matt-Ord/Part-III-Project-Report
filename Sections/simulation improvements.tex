\subsection{Q dependance}
TODO-
For the purpose of the simulation
it was necessary to assume a q independent
matrix element, \ldots to have a decent number of states \ldots
\ldots half interaction to give effective average

In the future it may also be possible to
directly investigate the effect
of q dependance in a much simpler system
with only one or two separate electron
energy states.
It would then be possible to select
several electron states randomly distributed
around a sphere of k states, and from
this an effective potential could be calculated.


Even in a q-independent model it may
be possible to find an improved effective
average overlap \ldots this could
provide around an additional
factor of 0.5 to the squared overlap

TODO- calculate tunnelling rate if half the
interaction potential.

TODO- Sparse matrix
TODO- Properly vectorise
TODO-

\subsection{Treatment of Electrons}
For the
However the true fermi energy, is slightly lower
at only \(1.24\times{} 10^{-18}J\). This
is possibly due to the two
valence electrons not being fully delocalised.
The true fermi surface
of Nickel is also not completely
spherical~\cite{FermiSufaceNickel},
and has a different level for each spin,
which leads to ferromagnetism~\cite{PhysRev.49.537}.

Since
only a small number of electrons
were included in each simulation
it would not be possible to include
such behaviour, however if we
perform the integration numerically
it should be possible to include these
in the lindblad model.

Ultimately however the effect is small \ldots


\subsection{Multiple Hydrogen Sites}
In the real Nickel lattice the FCC and
HCP sites are arranged in a regular
lattice, such that there are \ldots
neighbouring HCP site of each FCC
hydrogen. In future investigations
it should be therefore be
possible to include the
effect of a larger number of
hydrogen sites,

however
it is unclear how best to
incorporate `hopping' behaviour
of the hydrogen--- in the real
experiment the hydrogen becomes
localised onto a specific site
at later times. An effect we will
fail to see in any \ldots integration of
the schrodinger equation.

% \subsection{Additional Factors}
% \ldots in the derivation we have assumed
% that the \(N \rightarrow N' \)
% tunnelling rate should
% It is however possible that the
% Additional \(N'(N-1)\) factor
% suggested in \cref{eqn:two hop corrected rate}
% suppresses the rate by an exra
% factor of \(\frac{1}{4}\)
% we need more data to be able to say more than
% an order of magnitude approximation