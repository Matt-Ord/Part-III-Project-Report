\subsection{Degenerate tunnelling at 150K}


The curve was initially fitted to
\begin{equation}
    R(N, N) = \frac{R_0}{\cosh{(B(N - 0.5))}}
\end{equation}
with both the initial and final occupancy
equal to N. \ldots however although the rate is small it
would not fall to zero as \(N \rightarrow 0\).
Since we expect the rate to be proportional to the
number of electrons we add a prefactor of \(N(N-1)\)
to the expression, giving us
\begin{equation}
    R = \frac{N(N-1) R_0}{\cosh{(B(N - 0.5))}}
\end{equation}
Although this has \ldots little effect in exp \ldots
when plotting \(\ln{R}\) against \(N\) we can see a
clear reduction in the rate, consistent with this factor
at lower occupations.

TODO-plot, fill out missing \ldots.

The degenerate tunnelling rates

\subsection{Non Degenerate Tunnelling}
To recover the tunnelling rate for hydrogen of different
energy we need to transform the rate equation
into the form

\ldots Discussion of different ways to transform



