\subsection{Degenerate tunnelling at 150K}


The curve was initially fitted to
\begin{equation}
    R(N, N) = \frac{R_0}{\cosh{(B(N - 0.5))}}
\end{equation}
with both the initial and final occupancy
equal to N. \ldots however although the rate is small it
would not fall to zero as \(N \rightarrow 0\).
Since we expect the rate to be proportional to the
number of electrons we add a prefactor of \(N(N-1)\)
to the expression, giving us
\begin{equation}
    R = \frac{N(N-1) R_0}{\cosh{(B(N - 0.5))}}
\end{equation}
To see the effect of this factor
states with only
one electron or one hole must be included.
In this
case the previous
oscillatory behaviour is replaced
by a more steady decay, which was fitted to
\(\exp{(-at^2)}\). TODO-justify in figure Due to this new
method of fitting the data the
single particle rates appeared
offset from the rest of the data,
however their rates were consistent
with the \(N(N-1)\) factor TODO-fig.

Although this has \ldots little effect in exp \ldots
when plotting \(\ln{R}\) against \(N\) we can see a
clear reduction in the rate, consistent with this factor
at lower occupations.

TODO-plot, fill out missing \ldots.

The degenerate tunnelling rates

\subsection{Non Degenerate Tunnelling}
To recover the tunnelling rate for hydrogen of different
energy we need to transform the rate equation
into the form

Due to the arguments outlined in
\cref{app:combined tunnelling rates}
a general system with two tunnelling
rates has a combined rate
\begin{equation}
    R_{tot}(N\rightarrow{}N') = R(N\rightarrow{}N') + R_{tot}(N'\rightarrow{}N)
\end{equation}
therefore any guess at the corrected rate
must be written in this form, where the
behaviour of \(R(N\rightarrow{}N')\)
should match up with \(0.5 R_{tot}(N\rightarrow{}N)\)
at \(N' = N\).

\section{Temperature Dependance}
The correction should also be well behaved
at all temperatures. As we reduce the
temperature the hydrogen
energy becomes much greater than the
range of electrons exited above the fermi
surface, and we therefore expect the tunnelling
rate to be `frozen out' as \(T\rightarrow{}0\).
TODO-THis is not seen\ldots

Two other possible forms for the
tunnelling rate were suggested
\begin{alignat}{2}
    R(N\rightarrow{}N') & = 0.5\frac{\sqrt{N(N-1)N'(N'-1)} R_0}{\cosh{(B(N - 0.5))}} \\
    R(N\rightarrow{}N') & = 0.5\frac{4N(N-1)N'(N'-1) R_0}{\cosh{(B(N - 0.5))}}
\end{alignat}

We could also consider the corrected rate
\ldots two processes. We should therefore
have a rate dependent
Both of these had very little
effect on the tunnelling rate
at \(150K\).

\section{Corrected Results}
Overall we see the corrected results
have very little



\ldots Discussion of different ways to transform



