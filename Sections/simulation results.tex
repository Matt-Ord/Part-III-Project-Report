
From the previous investigation it is clear
that it is not possible to incorporate the
hydrogen energy directly \ldots. In the
real system the electrons would move
between states at a much shorter time
period than that of hydrogen tunnelling.
After an electron transitions to a lower
band the fermi distribution should therefore
be recovered almost instantaneously.
The tunneling rate ignoring the
hydrogen energy difference should
therefore provide a good approximation
for the true tunneling rate, as the
electron should only spend a small
fraction of it's time in the lower band.

\subsection{}
To calculate the overall tunneling rate
the simulation was repeated in small
bands (of width cite TODO)




\subsection{Degenerate Tunnelling at 150K}
The tunnelling was simulated at various occupancy,
however the rate was seen to change as the
number of electrons was increased.

and the rate curve was fitted to
\begin{equation}
    R(N,N) = 4 R_0 N(1-N)\label{eqn:degenerate tunnelling rate}
\end{equation}

\subsection{Tunnelling at Half Filling}
Since the rate depends only on a single
parameter \(R_0\) we are able to
extrapolate the rate at all
occupancy from the asymptotic
limit of the rate at \(N=0.5\).
Plotting \ldots TODO we find
a value of

TODO plot


\subsection{Complete Tunnelling Rate}
To recover the complete tunneling
rate the occupation rate curve
must first be converted into an
energy rate using the
fermi-dirac distribution (TODO CITE)
\begin{align}
    N(\epsilon)           & = \frac{1}{1 + \exp{(\beta(\epsilon - \mu))}}                                                      \\
    R(\epsilon, \epsilon) & = 4R_0 \frac{1}{1 + \exp{(\beta(\epsilon - \mu))}}(1- \frac{1}{1 + \exp{(\beta(\epsilon - \mu))}}) \\
                          & = 4R_0 \frac{\exp{(\beta(\epsilon - \mu))}}{{(1 + \exp{(\beta(\epsilon - \mu))})}^2}
\end{align}
where we have assumed uniform
occupation within a band.
The rate curve (TODO-figure) can then be
integrated to give an
overall tunneling rate of \ldots.
\begin{figure}
    TODO
\end{figure}

\subsection{Correcting for Reversed Rate}
In reality however the occupation
of the lower band should be larger than
that of the higher energy band. Due to
the arguments outlined in
\cref{app:combined tunnelling rates}
a general system with a different
forward and backward rate has a combined rate
\begin{equation}
    R(N,N') = R(N\rightarrow{}N') + R(N'\rightarrow{}N)
\end{equation}
where \(R(N\rightarrow{}N')\) is the
tunneling rate from an occupation \(N\)
to an occupation \(N'\). From
\cref{eqn:degenerate tunnelling rate}
it is clear that
\begin{equation}
    R(N\rightarrow{}N) = 2R_0N(1-N)
\end{equation}
By requiring the correct
behaviour at \(N' = N\), there are
three ways to modify the tunneling
rate
\begin{align}
    R(N\rightarrow{}N') & = 2R_0N(1-N)                 \\
    R(N\rightarrow{}N') & = 2R_0N(1-N')                \\
    R(N\rightarrow{}N') & = 2R_0 \sqrt{N(1-N)N'(1-N')}
\end{align}
note we have ignored functions
such as \(2R_0N'(1-N')\) which
would give the same overall rate
as \(2R_0N(1-N)\).

\section{Investigating Rate Corrections}

% \begin{equation}
%     \epsilon(N)= k_b T (\mu + \ln{\frac{1 - N}{N}} )
% \end{equation}




\subsection{Tunneling without Self Interaction}
\ldots only introduce an offset in energy
we might be tempted to exclude this self interaction

the tunnelling rate is however much different
in this case





\subsection{Simulation With Multiple Spins}


\subsection{Bosons with Constrained Occupation}
To investigate the effect of fermion spin
statistics \ldots to calculate the rate
for bosons.
