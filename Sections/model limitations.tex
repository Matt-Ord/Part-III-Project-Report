

\subsection{Q Dependance}
One of the key approximations
made in the model of the
Electron-Hydrogen system
was the q-independence of
the interaction hamiltonian.
Although this only contributed
to a variation of
\(14.5\% \) to the interaction
potential
(\cref{sec:electron hydrogen potential})
rapid oscillations
in the hydrogen overlap
fourier transforms
could mean that scattering
between certain wavevectors
should be excluded entirely.
It is therefore possible
that is approximation
is the cause of
the largest source of
error in our analysis.

For the purpose
of the simulation
it will always
be necessary to assume a
q independent
matrix element in general,
as it is not possible
to include enough
states to properly
sample all possible final
states in 3D.
We should however be able
to incorporate the
effects of a q-dependent
interaction into the
Lindblad analysis,
as it would only effect
the integral in \cref{eqn:gamma integral form}.
This analysis could then
be applied to the simulation,
producing an effective matrix
element which would only depend
on the energy difference between
two electron states.


\subsection{Realistic Treatment of Electrons}
In the development
of the Electron-Hydrogen
model we also assumed
an ideal electron gas,
with a fermi energy
of \(1.91\times{}10^{-18}J\).
The real fermi energy
of Nickel actually slightly lower,
at only \(1.24\times{} 10^{-18}J\),
possibly since the two
valence electrons are not
fully delocalised.
The fermi surface of Nickel
is also not entirely
spherical~\cite{FermiSufaceNickel},
indeed it is different
for each spin.
This is the cause of
ferromagnetic properties
within the metal~\cite{PhysRev.49.537}.

This behaviour could
be included in the
Lindblad analysis through
a modification of \cref{eqn:gamma integral form},
however it is not possible
to incorporate any
non-spherical behaviour
into the simulation
without the introduction of
an effective constant.
Overall we only expect
this to lead to a small
correction to the rate,
especially at finite temperature
when the surface is `smoothed out'
as electrons are excited out
of the groundstate.





