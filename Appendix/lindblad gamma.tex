Starting from the definition of
the Lindblad rate constant \(\gamma{}\)
(\cref{eqn:gamma definition}) and
the environment interaction hamiltonian
(\cref{eqn:split interaction hamiltonian})
we find
\begin{equation}
    \Gamma_{i,j, k,l}(\omega) =
    \int_0^\infty{}{
    ds \exp{(i\omega{}s)}
    Tr_{E}[E^\dagger_{i,j}(t)E_{k,l}(t-s)\rho_E(0)]
    }
\end{equation}
where
\begin{align}
    E_{i, j}(t) & =
    \exp{(iH_e t)}
    \sum_{k,k'} V_{i,j} \hat{b}^\dagger_{k',s'}\hat{b}_{k,s}
    \exp{(-iH_e t)}                           \\
                & = \sum_{k,k'} \hat{V}_{i,j}
    \hat{b}^\dagger_{k',s'}\hat{b}_{k,s} \exp{(i(E_k' - E_k)t)}
\end{align}
For a purely
statistical ensemble of electrons
the density matrix is
diagonal~\cite{sakurai_napolitano_2020}
\begin{equation}
    \rho_E(0) = \sum_{\{N(k)\}}
    P(\{N(k)\})
    \ket{N(k)} \bra{N(k)}
\end{equation}
where \(P(N(k)) =
\frac{1}{z}\sum_{k,s}
\exp{(-{N(k)}_s(\beta E_k - \mu))}\).

We start by expanding out the trace
over the environment
\begin{align}
    Tr_E[\dots] & = \sum_{\{N(k)\}}
    \bra{N(k)} E^\dagger_{i,j}(t)E_{k,l}(t-s) \rho_E(0) \ket{N(k)}
    \\
                & = \begin{aligned}[t]
        \sum_{\{N(k)\}, \{N'(k)\}} &
        P(\{N'(k)\}) \bra{N'(k)}  \ket{N(k)}                                               \\
                                   & \bra{N(k)} E^\dagger_{i,j}(t)E_{k,l}(t-s) \ket{N'(k)}
    \end{aligned} \\
                & = \sum_{\{N(k)\}}
    P(\{N(k)\}) \bra{N(k)}
    E^\dagger_{i,j}(t)E_{k,l}(t-s) \ket{N(k)} \\
                & = \begin{aligned}[t]
        \sum_{\substack{\{N(k)\}                             \\
        k_1,s^1,k_2,s^2                                      \\
                k_3,s^3,k_4,s^4 }}
         & P(\{N(k)\}) V_{i,j} V_{k,l}                       \\
         & \exp{(i(E_1 - E_2) t)} \exp{(i(E_3 - E_4) (t-s))} \\
         & \bra{N(k)}
        \hat{b}_{k_1,s^1}^\dagger{} \hat{b}_{k_2,s^2}
        \hat{b}_{k_3,s^3}^\dagger{} \hat{b}_{k_4,s^4}
        \ket{N(k)}
    \end{aligned}
\end{align}
Where here we have used
the fact that the
potential is real, and
swapped the order of \(k_1, k_2\)
from the usual definition.
This expression is non zero
in only two cases
\begin{itemize}
    \item \(k_1=k_2, s^1=s^2\),
          \(k_3=k_4, s^3=s^4\)
    \item \(k_1=k_4, s^1=s^4\),
          \(k_3=k_2, s^3=s^2\) but
          \(k_1\neq{}k_2, s^1\neq{}s^2\)
\end{itemize}
we use the result
\begin{align}
    \bra{N(k)}
    \hat{b}_{k_1,s^1}^\dagger{}
    \hat{b}_{k_1,s^1}
    \hat{b}_{k_3,s^3}^\dagger{}
    \hat{b}_{k_3,s^3}\ket{N(k)} & = N_1 N_3       \\
    \bra{N(k)}
    \hat{b}_{k_1,s^1}^\dagger{}
    \hat{b}_{k_3,s^3}
    \hat{b}_{k_3,s^3}^\dagger{}
    \hat{b}_{k_1,s^1}\ket{N(k)} & = \bra{N(k)}
    \hat{b}_{k_1,s^1}^\dagger{}
    \hat{b}_{k_1,s^1}
    \hat{b}_{k_3,s^3}
    \hat{b}_{k_3,s^3}^\dagger{}
    \ket{N(k)}                                    \\
                                & = N_1 (1 - N_3)
\end{align}
to simplify the above sum
\begin{equation}
    Tr_E[\dots] = \begin{aligned}[t]
        \sum_{k_1,s^1,k_3,s^3 }
         & V_{i,j} V_{k,l} [ \\
         & N_1 N_3
                + N_1 (1 - N_3) \exp{(-i(E_3 - E_1)s)}]
    \end{aligned}
\end{equation}
If we then integrate over s we find
\begin{align}
    \Gamma_{i,j, k,l}(\omega) & =
    \int_0^\infty{}{
        ds \exp{(i\omega{}s)} Tr_{E}[\dots]
    }                                                      \\
    {}                        & =\begin{aligned}[t]
        \sum_{k_1,s^1,k_3,s^3,k_4,s^4 }
         & V_{i,j} V_{k,l} [ \\
         & N_1 N_3 \delta(w)
                + N_1 (1 - N_3)  \delta(w + E_1 -E_3) ]
    \end{aligned}
\end{align}
to convert these delta functions
into delta function in momentum we
use the formula
\(\delta(f(x)) =
\abs{\frac{df}{dx}}^{-1}\delta(x - x_0)\)
to give
\begin{equation}
    \delta(w + E_1 -E_3) =
    \frac{m_e}{\sqrt{k_1^2 - 2m_e\omega}}
    \delta({k_3 \pm \sqrt{k_1^2 + 2m_e\omega}})
\end{equation}
Note we are working in units of \(\hbar = 1\)
and the \(\delta(\omega)\) is just
constraining \(E_3 = E_4\) which is
already satisfied when \(k_3 = k_4\).
Adding this back into the expression
for \(\Gamma \) we find
\begin{align}
    \Gamma_{i,j, k,l}(\omega) & =\begin{aligned}[t]
        \sum_{k_1,s^1,k_3,s^3 }
         & V_{i,j} V_{k,l} [
        N_1 N_3 \delta_{w, 0} \frac{m_e}{\sqrt{k_3^2}} \\
         & + N_1 (1 - N_3)
                \frac{m_e}{\sqrt{k_1^2 - 2m_e\omega}}
                \delta({k_3 \pm \sqrt{k_1^2 + 2m_e\omega}}) ]
    \end{aligned}
\end{align}
To calculate these terms we
need to switch to the integral
representation. Absorbing the factors
of \(L^6\) back into the definition
of \(V_{i,j}\) we find
\begin{align}
    \Gamma_{i,j, k,l}(\omega) & =\begin{aligned}[t]
        \sum_{s^1,s^3} \int &
        \frac{d^3\vec{k}_1}{{(2\pi)}^3}
        \frac{d^3\vec{k}_3}{{(2\pi)}^3}
        V_{i,j} V_{k,l} [
        N_1 N_3 \delta_{w, 0} \frac{m_e}{\sqrt{k_3^2}} \\
                            & + N_1 (1 - N_3)
                \frac{m_e}{\sqrt{k_1^2 - 2m_e\omega}}
                \delta({k_3 \pm \sqrt{k_1^2 + 2m_e\omega}}) ]
    \end{aligned}
\end{align}
we perform the integral over k by expanding about
\(k = k_f\) noting that the value of
\(\omega \) is equal to the energy
difference of the hydrogen
\(\Gamma_{i,j, k,l}(\omega) = \Gamma_{i,j, k,l}(\omega_{k,l})\)
where \(\omega_{k,l} = E_k - E_l\)
(see \cref{eqn:cross terms density matrix evolution}).
\begin{align}
    1 - N_3 & = \frac{1}{1 + \exp{(-\beta(E_3 - \mu))}}                                  \\
            & = \frac{1}{1 + \exp{(-\beta(E_1 + \omega - \mu))}}                         \\
            & \sim \frac{1}{2 + -\beta(E_1 + \omega - \mu)}                              \\
            & \sim \frac{1}{(1 - \frac{\beta \omega}{2})(2 + -\beta(E_1  - \mu))}        \\
            & \sim \exp{(\frac{\beta \omega}{2})}\frac{1}{1 + \exp{(-\beta(E_1 - \mu))}}
\end{align}

We then discard the first part of the integral by noting no
terms with \(\omega = 0\) appear in the
final expression for \(\dot{\rho}\), and note
\(\omega \ll k_1\) at the fermi surface.
\begin{align}
    \Gamma_{i,j, k,l}(\omega_{k,l}) & =\begin{aligned}[t]
        \sum_{s^1,s^3} \exp{(\frac{\beta \omega_{k,l}}{2})} \int &
        \frac{m_e{(4\pi)}^2 k_1^4 dk_1}{{(2\pi)}^6\sqrt{k_1^2 - 2m_e\omega}}
        V_{i,j} V_{k,l} [ N_1 (1 - N_1)]
    \end{aligned} \\
                                    & =\begin{aligned}[t]
        \sum_{s^1,s^3} \exp{(\frac{\beta \omega_{k,l}}{2})} \int &
        \frac{m_e k_1^3 dk_1}{4\pi^4}
        V_{i,j} V_{k,l} [ N_1 (1 - N_1)]
    \end{aligned}
\end{align}
we expand \(N_1 (1 - N_1)\) about \(k_1 = k_f\)
\begin{align}
    N_1 (1 - N_1) & = \frac{1}{1 + \exp{\beta \Delta E}}
    \frac{1}{1 + \exp{-\beta \Delta E}}                                                \\
                  & \sim \frac{1}{2 + \beta \Delta E + \frac{{(\beta \Delta E)}^2}{2}}
    \frac{1}{2 - \beta \Delta E + \frac{{(\beta \Delta E)}^2}{2}}                      \\
                  & = \frac{1}{4}
    \frac{1}{1 + \frac{\beta \Delta E}{2} + \frac{{(\beta \Delta E)}^2}{4}}
    \frac{1}{1 - \frac{\beta \Delta E}{2} + \frac{{(\beta \Delta E)}^2}{4}}            \\
                  & \sim \begin{aligned}[t]
        \frac{1}{4}
         & (1 - \frac{\beta \Delta E}{2} - \frac{{(\beta \Delta E)}^2}{4} + {(\frac{\beta \Delta E}{2})}^2) \\
         & (1 + \frac{\beta \Delta E}{2} - \frac{{(\beta \Delta E)}^2}{4} + {(\frac{\beta \Delta E}{2})}^2)
    \end{aligned}                                    \\
                  & = \frac{1}{4}(1 - \frac{{(\beta \Delta E)}^2}{4})                  \\
                  & \sim \frac{1}{4}\exp{(- \frac{{(\beta \Delta E)}^2}{4})}
\end{align}
The sharp exponential decay justifies
this expansion. Going back to the
integral we have
\begin{align}
    \Gamma_{i,j, k,l}(\omega_{k,l}) & =\begin{aligned}[t]
        \sum_{s^1,s^3} \exp{(\frac{\beta \omega_{k,l}}{2})} \int &
        \frac{m_e k_1^3 dk_1}{{(2\pi)}^4}
        V_{i,j} V_{k,l} \exp{(- \frac{{(\beta \Delta E)}^2}{4})}
    \end{aligned} \\
                                    & =\begin{aligned}[t]
        \sum_{s^1,s^3} \exp{(\frac{\beta \omega_{k,l}}{2})} \int &
        \frac{m_e k_1^3 dk_1}{{(2\pi)}^4}
        V_{i,j} V_{k,l} \exp{(- \frac{{(\frac{\beta \hbar^2}{2 m_e}(k_1^2 - k_f^2))}^2}{4})}
    \end{aligned} \\
                                    & =\begin{aligned}[t]
        \sum_{s^1,s^3} \exp{(\frac{\beta \omega_{k,l}}{2})} \int &
        \frac{m_e k_f^2 du}{2{(2\pi)}^4}
        V_{i,j} V_{k,l} \exp{(- \frac{\beta^2 \hbar^4}{{(4m_e)}^2} u^2)}
    \end{aligned} \\
                                    & =\begin{aligned}[t]
        \sum_{s^1,s^3} \exp{(\frac{\beta \omega_{k,l}}{2})} \frac{m_e k_f^2 }{{(2\pi)}^4}
        V_{i,j} V_{k,l} \sqrt{\pi} \frac{2m_e}{\beta \hbar^2}
    \end{aligned}
\end{align}
checking the units of \(\Gamma \) we find
\begin{align}
    [\Gamma] & = {[kg]}^2[m^{-2}]{[kgm^5s^{-2}]}^2{[kgm^2s^{-2}]}^{1}{[kg m^2 s^{-1}]}^{-2} \\
             & = [{kg}^2 m^{-2} m^{6} s^{-2} {kg}^{1} m^{2} s^{-2}]                         \\
             & = [{kg}^3 m^{6}s^{-4}]
\end{align}
to recover the correct units of \(s^{-1}\) we
therefore divide by \(\hbar^3 \) which was previously
taken to be 1. We can also write
\(V_{i,j}
= \mathcal{C}_{i,j} \frac{2e^2}{\epsilon_0 \alpha^2}
= \mathcal{C}_{i,j} \frac{8 \pi^2 \epsilon_0 \hbar^4}{e^2 m_e^2}\)
where \(\mathcal{C}_{i,j}\) is the hydrogen
overlap factor given in \cref{sec:simplified interaction}
\begin{align}
    \Gamma_{i,j, k,l}(\omega_{k,l}) & =
    \sum_{s^1,s^3} \exp{(\frac{\beta \omega_{k,l}}{2})}
    \frac{m k_f^2 }{{(2\pi)}^4}
    \mathcal{C}_{i,j} \mathcal{C}_{k,l}
    \sqrt{\pi} \frac{2m}{\beta \hbar^5} {(\frac{8 \pi^2 \epsilon_0 \hbar^4}{e^2 m_e^2})}^2 \\
                                    & =
    \sum_{s^1,s^3} \exp{(\frac{\beta \omega_{k,l}}{2})}
    \mathcal{C}_{i,j} \mathcal{C}_{k,l}
    \sqrt{\pi} \frac{8 k_f^2 \epsilon_0^2 \hbar^3}{\beta e^4 m_e^2}
\end{align}

If we go back to calculate the
\(\omega = 0\) contribution
we find (by taking the low
temperature limit and noting
particle number is constant
with temperature)
\begin{align}
    \Gamma_{i,j, k,l}(\omega) & =
    \sum_{s^1,s^3} \int_0^{k_f}
    \frac{d^3\vec{k}_1}{{(2\pi)}^3}
    \frac{d^3\vec{k}_3}{{(2\pi)}^3}
    V_{i,j} V_{k,l} [
    N_1 N_3 \delta_{w, 0} \frac{m_e}{k_3}] \\
                              & =
    \sum_{s^1,s^3} \int_0^{k_f}
    \frac{4\pi k_1^2 dk_1}{{(2\pi)}^3}
    \frac{4 \pi m_e k_3 dk_3}{{(2\pi)}^3}
    V_{i,j} V_{k,l} \delta_{w, 0}          \\
                              & =
    \sum_{s^1,s^3} \frac{4\pi k_f^3 m_e}{3{(2\pi)}^3}
    \frac{2 \pi k_f^2}{{(2\pi)}^3}
    V_{i,j} V_{k,l} \delta_{w, 0}          \\
                              & =
    \sum_{s^1,s^3} \frac{4\pi k_f^3 m_e}{3{(2\pi)}^3}
    \frac{2 \pi k_f^2}{{(2\pi)}^3}
    V_{i,j} V_{k,l} \delta_{w, 0}          \\
\end{align}
If we add back in the factor of  \(\hbar^3\)
\begin{align}
    \Gamma_{i,j, k,l}(0) & =
    \sum_{s^1,s^3} \frac{4\pi k_f^3 m_e}{3\hbar^3{(2\pi)}^3}
    \frac{2 \pi k_f^2}{{(2\pi)}^3}
    \mathcal{C}_{i,j} \mathcal{C}_{k,l}
    \delta_{w, 0} {(\frac{8 \pi^2 \epsilon_0 \hbar^4}{e^2 m_e^2})}^2 \\
                         & =
    \sum_{s^1,s^3}
    \frac{8k_f^5 \epsilon_0^2 \hbar^5}{3e^4 m_e^3}
    \mathcal{C}_{i,j} \mathcal{C}_{k,l} \delta_{w, 0}
\end{align}
If we use dimensional analysis on this term
we find it is proportional to \(m^{-1}s^{-1}\).
This is because we have ignored an important
subtlety when imposing the \(\omega \) delta
function. If we had done it properly (before
setting \(k_3 = k_4\)) we would get
out an extra factor of \(L\). This term
therefore diverges as \(L \rightarrow \infty \).
\begin{align}
    \Gamma_{i,j, k,l}(0) & =
    L \sum_{s^1,s^3}
    \frac{8k_f^5 \epsilon_0^2 \hbar^5}{3e^4 m_e^3}
    \mathcal{C}_{i,j} \mathcal{C}_{k,l} \delta_{w, 0}
    \label{eqn:divergent expression for first integral}
\end{align}