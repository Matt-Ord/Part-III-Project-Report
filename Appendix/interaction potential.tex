The potential of hydrogen
in the position basis can be found
making use of the greens
function of the coulomb potential\cite{AQP_Problems}
\begin{equation}
    V(\vec{r}) = \frac{e^2}{4 \pi \epsilon_0}(
    -\frac{1}{r}
    + \int{\frac{\abs{\psi(\vec{r})}^2}{
            \abs{\vec{r} - \vec{r'}}} d\vec{r}'})
\end{equation}
when we assume the electron
lies in the 1s state\cite{griffiths_schroeter_2018}
\begin{equation}
    \psi(\vec{r}) = {(\pi a_0^3)}^{\frac{1}{2}} e^{-\frac{r}{a_0}}
\end{equation}
we arrive at the following potential
\begin{equation}
    V(\vec{r}) = \frac{e^2}{4 \pi \epsilon_0}(
    -\frac{1}{r}
    + \pi a_0^3\int{\frac{e^{-\frac{2r}{a_0}}}{
            \abs{\vec{r} - \vec{r'}}} d\vec{r}'})
\end{equation}
If we make use of two standard results
\begin{align}
    \int{\frac{1}{r} e^{i\vec{q}.\vec{r}} d^3\vec{r}}
     & = \frac{4 \pi}{q^2}                         \\
    \int{e^{-\alpha r} e^{i\vec{q}.\vec{r}} d^3\vec{r}}
     & = \frac{8 \pi \alpha}{{(\alpha^2 + q^2)}^2} \\
\end{align}
with \(\alpha = \frac{2}{a_0}\) we end up
with
\begin{align}
    V(\vec{q}) & = \frac{e^2}{4 \pi \epsilon_0}(
    - \frac{4\pi}{q^2}
    + \frac{\pi a_0^3}{q^2}
    \frac{8 \pi \alpha}{{(\alpha^2 + q^2)}^2})   \\
               & = \frac{e^2}{\epsilon_0 q^2}(
    \frac{ \alpha^4}{{(\alpha^2 + q^2)}^2} - 1
    )                                            \\
               & =-\frac{e^2}{\epsilon_0 }
    \frac{ 2\alpha^2 + q^2 }{{(\alpha^2 + q^2)}^2}
\end{align}