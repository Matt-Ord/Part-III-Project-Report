\documentclass{article}
\usepackage[utf8]{inputenc}

\usepackage{mathtools}
\usepackage{braket}
\usepackage{amssymb}

\usepackage{cleveref}
\usepackage{biblatex}
\addbibresource{main.bib}

\usepackage{caption}
\usepackage{subcaption}

\usepackage{booktabs}

% vectors bold not underlined
\renewcommand{\vec}[1]{\mathbf{#1}}
% abs symbol
\DeclarePairedDelimiter\abs{\lvert}{\rvert}

\title{TODO Name Project}
\author{Matthew Ord}
\date{}

\begin{document}
\maketitle
The main text (excluding appendices and abstract) should be concise (20–30 pages, 5000 words maximum (excluding references))
4170 out of 5000
\begin{abstract}
    This is the abstract
\end{abstract}



\pagebreak
\section{Introduction}

% How does it fit into the bigger picture

% Numerical solver of Lindblad/ Redfield
% equation http://qutip.org/docs/latest/guide/dynamics/dynamics-bloch-redfield.html
The aim of this report is to
investigate the extent to which
a first-principles calculation
could be used to predict the rate of
incoherent ground state to
ground state tunneling of hydrogen
on the surface of nickel.

Hydrogen diffusion has been measured
on the surface of Ni(111) %ChkTeX 36
through many different
techniques\cite{LIN199141, Ni_Diffusion_Experement},
most recently
using Helium-3 spin-echo
interferometry\cite{Helium_spin_echo}.
With the help of first-principle DFT calculations of
the H-Ni(111) potential a %ChkTeX 36
quantum version of transition state theory (ASTST)
was developed to describe these tunneling rates\cite{Jianding-Zhu}.
The results of this theory
were found to closely
match the predicted
hopping rates, particularly
when the rate was dominated
by activated tunnelling\cite{Jianding-Zhu}.
There was however a difference
in the rate at low temperatures,
where it is thought that interaction with the
electron gas leads to incoherent tunnelling.
This type of interaction
is of particular relevance in
the development of superconducting interference devices
(SQUIDs)\cite{QubitIncoherentSaito2004}
and has been successfully described
through the introduction
of effective dissipative models\cite{CALDEIRA1983374}.
To be able to assess this approach
however it is first necessary to
investigate the propagation of hydrogen directly
by introducing a model of the
complete electron-hydrogen
interaction.

To find an effective
theory describing only the system
we can trace out the
environmental degrees of
freedom to produce a quantum master
equation. Initial
attempts to find such an
equation by Redfield\cite{REDFIELD19651}
and Lamb\cite{PhysRev.134.A1429} produce
many useful results, however in
general they do not preserve the
trace of the density matrix\cite{Chru_ci_ski_2017}. This
issue was eventually solved through
the introduction of the
Lindblad equation, developed
by Lindblad\cite{Lindblad1976}
and simultaneously
by V. Gorini et al.~\cite{doi:10.1063/1.522979}.



A model for the electron-hydrogen
interaction is outlined in
\cref{sec:the model},
which is used
alongside the
Lindblad equation
to find an analytical prediction
of the tunnelling rate
in \cref{sec:redfield}.
In
\cref{sec:simulation,sec:simulation results}
the behaviour of the
system was also
investigated
through direct integration
of the
Schrödinger equation,
including both the dynamics
of the system and the environment.
The similarities and
limitations of the two
methods are discussed
in \cref{sec:improvements}
before the conclusions of
the report are
outlined in \cref{sec:conclusion}.






\section{Lit review}
% Outline the state of the art

\section{Theory}\label{sec:the model}
Explain what the equations mean
\subsection{The Hamiltonian}
The \ldots was modeled using a simple
hamiltonian, assuming

\ldots system of electrons interacting with
a single hydrogen atom which would lie at
either a low or a high energy site
\begin{equation}
    \hat{H} = \hat{H}_{free} + \hat{H}_{int}
\end{equation}
where the free hamiltonian is given by
\begin{eqnarray}
    \hat{H}_{free} &=& \hat{H}_{e^-} + \hat{H}_{h}\\
    \hat{H}_{e^-} &=& \sum_{k, s}
    \frac{\hbar^2 k^2}{2m_e} \hat{b}^\dagger_{k, s} \hat{b}_{k, s}\\
    \hat{H}_{h} &=&
    E_0 \hat{a}^\dagger_0 \hat{a}_0
    + E_1 \hat{a}^\dagger_1 \hat{a}_1
\end{eqnarray}
\(\hat{a}^\dagger_i\) is the hydrogen creation
operator for the site i, and satisfies the standard commutation
relations for a boson \(\left[ \hat{a}_i, \hat{a}^\dagger_j \right] = \delta_{ij}\),
and \(\hat{b}^\dagger_{k, s}\) is the electron creation operator
with spin \(s\) and wavevector \(k\) satisfying the standard
fermion anti-commutation relations
\( \{ \hat{b}_{k, s}, \hat{b}^\dagger_{k', s'} \} = \delta_{k k'} \delta_{s s'}\).

TODO- SHOW THIS
\ldots giving us the
\begin{equation}
    \hat{H}_{int} = \frac{1}{L^3}\sum_{k,s,k',s'}\hat{b}^\dagger_{k',s'}\hat{b}_{k,s}\tilde{V}(\vec{q})\int{d\vec{r}
    \hat{\psi}_h^{\dagger}\hat{\psi}_h \exp(i\vec{q}.\vec{r})}
\end{equation}
We then gather the q dependant terms into
an effective potential
\begin{equation}
    \hat{H}_{int} = \sum_{k,s,k',s',i,j}
    {\tilde{V}_{eff}(\vec{q})}_{i,j}
    \hat{b}^\dagger_{k',s'}\hat{b}_{k,s}
    \hat{a}^\dagger_{i}\hat{a}_{j}
    \label{eqn:interaction hamiltonian in k}
\end{equation}

\subsection{The Electron Hydrogen Potential}

The electron surrounding the hydrogen was assumed
to lie in the 1s orbital
TODO-SHOW ENERGY TO DISSOCIATE IS TOO LARGE

The potential is then given by the equation
\begin{equation}
    V(\vec{r}) = \frac{e^2}{4 \pi \epsilon_0}(
    -\frac{1}{r}
    + \int{\frac{\abs{\phi(\vec{r}')}^2}{
            \abs{\vec{r} - \vec{r'}}} d\vec{r}'})
\end{equation}
where
\begin{equation}
    \phi(\vec{r}) = {(\pi a_0^3)}^{-1/2} e^{-\frac{r}{a_0}}
\end{equation}
is the 1s hydrogen orbital, and \(a_0\) is
the bohr radius. We then fourier
transform this expression
(see \cref{app:interaction potential calculation})
to find
\begin{eqnarray}
    V(\vec{q}) &=& \frac{e^2}{\epsilon_0 q^2}(
    \frac{\alpha^4}{{(\alpha^2 + q^2)}^2} - 1
    )
\end{eqnarray}
with \(\alpha = \frac{2}{a_0}\). If we assume scattering
occurs between states separated by at most
twice the fermi energy \(q \leq 2k_f\) we find
TODO- JUSTIFY Q=0

\subsection{The Hydrogen Wavefunction}
TODO-
The form of the hydrogen wavefunctions were
known from previous

\ldots using these

To produce a localised wavepacket several of these
sates are chosen around \(q=0\)

It was then possible to calculate the fourier
transform directly

The calculation gave

Although the fourier transform varied was not constant
the rate
TODO-JUSTIFY

\section{Simulation Investigation}\label{sec:simulation}
how to setup the experiment
\subsection{Eigenvalue Decomposition}
The dynamics of the electron-hydrogen
system discussed in \cref{sec:the model}
can be investigated by
directly simulating the system.
This can be done through direct
integration of the
Schrödinger equation, however
if we first decompose
the initial state
into eigenstates
of the complete hamiltonian\cite{conduit}
\begin{align}
    \ket{\Psi(t)} = \exp{(-i\frac{Ht}{\hbar})} \sum_n C_n \ket{n} \\
    = \sum_n C_n \exp{(-i\frac{E_n t}{\hbar})} \ket{n}
\end{align}
we can propagate
by multiplying each eigenstate
by a phase-factor.

Since we are dealing with a
large number of eigenstates
it is important to think
about both the storage
and computational complexity
of the two methods (\cref{tab:algorithm complexity}).
\begin{table}[htbp]
    \begin{center}
        \begin{tabular}{ *{3}{c} }
            \toprule
            Cost    & Integration            & Decomposition              \\
            \midrule
            Time    & \(\mathcal{O}(n^2 t)\) & \(\mathcal{O}(n^3 + n d)\) \\
            Storage & \(\mathcal{O}(n d)\)   & \(\mathcal{O}(n^2 + n d)\) \\
            \bottomrule
        \end{tabular}
    \end{center}
    \caption{Complexity associated with the
        two methods of solving Schrödinger equation,
        where \(n\) is the number of eigenstates, t
        is the number of timesteps and d is the
        number of datapoints. The method
        of eigenvalue decomposition
        prevents the \(\mathcal{O}(t)\)
        dependence seen
        in direct integration
        by first decomposing the
        eigenstate (\(\mathcal{O}(n^3)\))
        before multiplying
        by the relevant phase
        (\(\mathcal{O}(nd)\)).
    }\label{tab:algorithm complexity}
\end{table}

As we
are only interested in the
evolution of the eigenstates
at times much greater than
the frequency of the sates
\(\omega = \frac{E}{\hbar}\)
the method of integration
was found to be much
slower than that of eigenvalue
decomposition. One
issue with this method
is the increased storage
cost associated with
the complete
hamiltonian. This could
be prevented by
working with a sparse
matrix, however in
practise this
was not required.

Although the decomposition is
expensive it was only repeated once,
which allowed us to gather a large
number of times at very little additional
cost. It was also possible to use the
fact that the matrix was hermitian to
provide and additional increase in speed.

\subsection{Matrix Representation}\label{sec:state representation}
Working in the unperturbed electron basis
(\cref{sec:electron states})
we label each eigenstate
according to the index of the
hydrogen site and
the configuration of the
electron system.
In general for \(n\)
electron states
and \(m\) hydrogen states
there would
be \(m 2^n\)
possible configurations,
however since the hamiltonian
conserves particle number
we limit ourselves
to a fixed
number of electrons (\(N\)).
In this case the number
of states scales as \(m \times{} \binom{n}{N}\).
In practise we are able to
simulate a system with
around \(3500\) eigenstates,
or \(14\) half filled electron states
(\cref{tab:number of eigenstates}).
\begin{table}[htbp]
    \begin{center}
        \begin{tabular}{ *{4}{c} }
            \toprule
            Number of States & All Configurations & Half Filled & 2 electrons \\
            \midrule
            \(10\)           & \(1024\)           & \(252\)     & \(45\)      \\
            \(12\)           & \(4096\)           & \(924\)     & \(66\)      \\
            \(14\)           & \(16384\)          & \(3432\)    & \(91\)      \\
            \(16\)           & \(65536\)          & \(12870\)   & \(120\)     \\
            \bottomrule
        \end{tabular}
    \end{center}
    \caption{
        Number of eigenstates required
        to store an electron system.
        Limiting ourselves to
        configurations with a fixed
        number of electrons we are
        able to simulate a larger
        number of states.
        This method scales particularly
        well for a system with a
        low number of electrons
        or a low number of holes.
    }\label{tab:number of eigenstates}
\end{table}

When working with
fermions we need to take
care over the exchange
statistics of the hamiltonian.
For self consistency we work in
a basis such that electrons
with lower energy are always
added first. Terms in the interaction
hamiltonian therefore pick
up a minus sign when there is
an odd number of electrons
between the exchanged energy levels.
\begin{align}
    a^\dagger_1a_2 \ket{2,3} & = \ket{1,3}  \\
    a^\dagger_1a_3 \ket{2,3} & = -\ket{1,2}
\end{align}

\subsection{Choosing the Initial States}

\subsubsection{Distribution Of Energies}
The distribution of electron energies was
initially chosen using an even spacing,
however this lead noise caused
by rabi oscillations
at a frequency
fixed through several runs
of the simulation.
To remove these oscillations
random offsets were introduced
into the energy distribution
which changed the rabi frequency
between runs,
allowing for cancellation
of the noise.

The spacing of electron energies is also
important for the simulation, as it
sets the effective volume of the nickel
lattice. The density of states of a free electron
gas \(g(E)\) is given by~\cite{KittelCharles1953Itss}
\begin{equation}
    g(E) = \frac{V}{2\pi^2}
    {(\frac{2m}{\hbar^2})}^{\frac{3}{2}}
    E^{\frac{1}{2}}
\end{equation}
We therefore invert this expression
to find the implied volume of the
simulation
\begin{equation}
    V = 2\pi^2
    \frac{g(E)}{E^{\frac{1}{2}}}
    {(\frac{\hbar^2}{2m})}^{\frac{3}{2}}
\end{equation}
Since at large \(k\)
(for \(k\sim k_f\))
the density of states is roughly
constant we make the approximation
\begin{equation}
    g(E) = \frac{dN}{dE} \sim \frac{1}{E_{space}}
\end{equation}
where \(E_{space}\) is the energy spacing
of the simulation. From \cref{eqn:simplified interacton potential} we find
\begin{equation}
    \hat{H}_{int} \propto \frac{1}{V} \propto E_{space}
    \label{eqn:energy spacing dependance of interaction hamiltonian}
\end{equation}
for smaller
energy spacing we have
a larger volume, and a smaller
perturbation.

\subsubsection{Distribution Of Electrons}
To be able to average over successive
simulations we also need to setup the
simulation with a somewhat random
choice of initial states.
We can start the hydrogen
in the FCC site by setting the
amplitudes of the HCP sites to zero
before choosing the FCC aptitudes according to
a normal distribution.

In simulations for which we choose states
with a large range of energies however
the electron distribution should not be
random and should follow the fermi dirac distribution.
To match this distribution we
need to alter the averages used
to produce the initial state vector.
Since the amplitude corresponds to
the square-root of the probability we
take the average magnitude as the
square-root of the boltzmann probability
associated with each state.
\begin{align}
    P_k & = \exp(-\beta{}E_k)     \\
    C_k & = \exp(-\beta{}E_k / 2)
\end{align}
where \(C_k\) is the wavefunction amplitude.
If we plot the resulting
electron distribution produced
by this method for fixed N
we see the expected electron
distribution for a standard electron
gas (\cref{fig:correct fermi dirac}).
\begin{figure}[htbp]
    \centering
    \begin{subfigure}{0.45\linewidth}
        \centering
        \includegraphics[width =0.9 \linewidth]{Figures/Simulation/Plot of correct fermi dirac distribution on center.png}
        \caption{5 Electrons 10 States}
    \end{subfigure}
    \hfill
    \begin{subfigure}{0.45\linewidth}
        \centering
        \includegraphics[width = 0.9\linewidth]{Figures/Simulation/Plot of correct fermi dirac distribution off centre.png}
        \caption{3 Electrons 10 States}
    \end{subfigure}
    \caption{Plot of the electron distribution seen
        when setting up the system randomly. The
        correct fermi-dirac distribution is seen in both the on and
        off center systems. Errors are given by the standard
        deviation of the electron densities, and are therefore
        larger around the fermi surface where fluctuations are
        large.}\label{fig:correct fermi dirac}
\end{figure}
However if we were to include
a large interaction term, such
as those required for the
real nickel system the
electron distribution is
seen to diverge, as the approach
does not account for the
shift in the perturbed
energies (\cref{fig:incorrect fermi dirac}).
\begin{figure}[htbp]
    \centering
    \begin{subfigure}{0.45\linewidth}
        \centering
        \includegraphics[width =0.9 \linewidth]{Figures/Simulation/Plot of incorrect fermi dirac distribution on center.png}
        \caption{Large interaction}
    \end{subfigure}
    \hfill
    \begin{subfigure}{0.45\linewidth}
        \centering
        \includegraphics[width = 0.9\linewidth]{Figures/Simulation/Plot of incorrect fermi dirac distribution on center small interaction.png}
        \caption{Reduced Interaction}\label{sub@fig:reduced interaction fermi-dirac}
    \end{subfigure}
    \caption{Plot of the fermi-dirac distribution
        with the inclusion of interaction. The
        correct distribution is no longer seen when
        the full interaction is included, however
        if this is reduced by a factor of \(10\)
        (\cref{sub@fig:reduced interaction fermi-dirac})
        the correct distribution is recovered.}\label{fig:incorrect fermi dirac}
\end{figure}
To overcome this limitation
the simulation was repeated
for small energy bands (\cref{sec:small band approach})
for which the electron distribution
was uniform.

\subsection{Initial Investigation}
To obtain a rough estimate of
the rate the system was setup
with electrons evenly spaced
in the region \(E_k = E_f \pm 2K_b T\)
with degenerate hydrogen energies (\cref{fig:tunnelling rate single large band}).
The hydrogen occupation
could then be inferred by
summing the occupation
of electrons in both the FCC and
HCP sites.
\begin{figure}[htbp]
    \captionsetup[subfigure]{justification=centering}
    \centering
    \begin{subfigure}{0.45\linewidth}
        \includegraphics[width=0.9\linewidth]{Figures/Simulation/Plot of large band simulation decay times.png}
        \subcaption{Tunnelling excluding hydrogen energy}\label{fig:large band degenerate simulation}
    \end{subfigure}
    \begin{subfigure}{0.45\linewidth}
        \includegraphics[width=0.9\linewidth]{Figures/Simulation/Plot of large band simulation decay times rapid oscillations.png }
        \subcaption{Rapid oscillation of the hydrogen occupation}
    \end{subfigure}
    \caption{Plot of the tunnelling rate taken using a
        simple choice of electron energies,
        taken evenly in the range \(E=E_f \pm 2 K_b T\).
        Simulation with a degenerate hydrogen
        (\cref{fig:large band degenerate simulation})
        shows tunnelling in around
        \(10^{-9}s\).
    }\label{fig:tunnelling rate single large band}
\end{figure}
Unfortunately we find that the interaction is
relatively large. The diagonal terms in
the interaction hamiltonian
had a value of \(6.21\times{}10^{-21}J\)
and a cross diagonal value of \(2.73\times{}10^{-23}J\)
whereas the electron energies were separated
by only \(9.2\times{}10^{-22}J\). This meant that
the perturbation approximation required for
fermi-dirac distributed states was not valid.
Once the hydrogen energy was included
we no longer see any tunneling
(\cref{fig:issues with single large band}).
\begin{figure}[htbp]
    \captionsetup[subfigure]{justification=centering}
    \centering
    \begin{subfigure}{0.45\linewidth}
        \includegraphics[width=0.9\linewidth]{Figures/Simulation/Plot of large band simulation with hydrogen energies.png}
        \subcaption{Tunnelling including hydrogen energy}\label{fig:large band non degenerate simulation}
    \end{subfigure}
    \begin{subfigure}{0.45\linewidth}
        \includegraphics[width=0.9\linewidth]{Figures/Simulation/Plot of large band electron distribution.png}
        \subcaption{Electron Distribution}\label{fig:large band fermi-dirac}
    \end{subfigure}
    \caption{Plot demonstrating the issues with the
        simple approach used to calculate the rate.
        When including the hydrogen energies
        (\cref{fig:large band non degenerate simulation})
        no tunnelling can be seen, even
        at much larger timescales. Even without the
        inclusion of hydrogen energies
        the incorrect electron distribution is
        seen (\cref{fig:large band fermi-dirac}).
    }\label{fig:issues with single large band}
\end{figure}

\subsection{Small Band Approach}\label{sec:small band approach}
To limit the effect of the incorrect
electron distribution we can
take electrons localised
in a small region of the fermi
surface such that their energies
and occupations are all similar.

One issue is that states at the edge
of the band will be `missing'
states to mix with during the
perturbation. We can approximate
this overlap using first order perturbation theory.
\begin{equation}
    \ket{n} = \ket{n^{(0)}} + \sum_{K\neq{}n} \frac{\bra{k^{(0)}} \hat{H}_{int} \ket{n^{(0)}}}{E_n^{(0)} - E_k^{(0)}} \ket{k^{(0)}}
\end{equation}
Since \(\hat{H}_{int} \propto E_{space}\)
(\cref{eqn:energy spacing dependance of interaction hamiltonian})
the degree of overlap of neighbouring states does
not depend on the energy spacing. Luckily
however the interaction is small
enough such that states only interact with
their closest neighbours (\cref{fig:single band energies}),
and the effect of these missing states should be small.
\begin{figure}[htbp]
    \centering
    \begin{subfigure}{0.45\linewidth}
        \includegraphics[width=0.9\linewidth]{Figures/Simulation/Plot of single band eigenstate energy range.png}
        \subcaption{Final State Energies}\label{fig:initial and final state energies of single band}
    \end{subfigure}
    \begin{subfigure}{0.45\linewidth}
        \includegraphics[width=0.9\linewidth]{Figures/Simulation/Plot of single band eigenstate energy range closeup.png}
        \subcaption{Detailed view of lower energies}\label{fig:initial and final state energies of single band zoom}
    \end{subfigure}
    \caption{Plot of average initial state energy weighted by the
        probability of occupation. The final
        state energy is separated into two,
        due to the symmetric and antisymmetric
        contributions
        (\cref{fig:initial and final state energies of single band}).
        In \cref{fig:initial and final state energies of single band zoom}
        we can see each state is highly degenerate, and mixing
        is limited to the four nearest neighbours.
    }\label{fig:single band energies}
\end{figure}


The tunneling could also be
influenced by these states
through a combination of
several electron `hops'.
Due to the energy time uncertainty
principle \(\Delta{}E\Delta{}T \geq \frac{\hbar}{2}\)
we expect the total energy to be conserved
to within
\begin{equation}
    \Delta{}E \sim \frac{\hbar}{2\Delta{} t}
\end{equation}
so that for a tunnelling time of
\(\sim 10^{-9}s\)
we expect an energy fluctuation
of \(\sim 5\times{}10^{-26} J\).
We therefore choose states separated
by at-least \(10^{-25} J\) such that
interaction with states outside
the band is prohibited
through energy conservation.


\subsection{Different Hydrogen Energy}\label{sec:different hydrogen energy}
If we add the hydrogen energies to the
simulation we no longer see tunnelling
on any timescale. Repeating the analysis
in \cref{sec:small band approach} we
see that eigenstate mixing and hence
tunnelling is dominated by
eigenstates degenerate in energy.
In the initial approach we therefore
miss many of the states which could
contribute to tunnelling. To ensure that we always have
states degenerate in energy we
therefore introduce two electron
bands separated by the hydrogen energy
difference.
\begin{figure}[htbp]
    \centering
    \begin{subfigure}{0.45\linewidth}
        \includegraphics[width=0.9\linewidth]{Figures/Simulation/single band eigenstate energies.png}
        \subcaption{One Band Overlaps}\label{fig:one band overlap}
    \end{subfigure}
    \begin{subfigure}{0.45\linewidth}
        \includegraphics[width=0.9\linewidth]{Figures/Simulation/two band eigenstate energies.png}
        \subcaption{Two Band Overlaps}\label{fig:two band overlap}
    \end{subfigure}
    \caption{Plot of the energies of the
        eigenstates, against the portion of the
        state which lies in the FCC site.
        For the single band approach we
        see very little overlap between the FCC and HCP
        initial states (\cref{fig:one band overlap})
        as the energies of the FCC and HCP sites
        are poorly matched. By introducing
        two bands separated by the hydrogen energy
        difference (\cref{fig:two band overlap})
        we find mixing between states which are
        degenerate in energy
    }\label{fig:overlap with hydrogen energies}
\end{figure}

If we were to plot the initial and final
state energies for the single band approach
we no longer see two distinct levels. Instead
we see three levels split into
several distinct groups.
To promote interaction between each group
we could increase the width of each band,
however this leads to a reduction in
the overlap between states. This is
caused by different mixing
within each band which prevents
the energy degeneracy. To solve this
we could simply remove the diagonal
interaction, however in (\cref{sec:tunnelling no diagonal})
we find this effects the
measured tunneling times.
Even when
working with a small band
we do not see the desired equilibrium
behaviour, calculated assuming the rate
scales as \(N(1-N)\).
\begin{figure}[htbp]
    \centering
    \begin{subfigure}{0.45\linewidth}
        \includegraphics[width=0.9\linewidth]{Figures/Simulation/Two Large Bands No Mixing.png}
        \subcaption{Large Band Mixing}\label{fig:two band minimum mixing}
    \end{subfigure}
    \begin{subfigure}{0.45\linewidth}
        \includegraphics[width=0.9\linewidth]{Figures/Simulation/Two Large Bands No Diagonal.png}
        \subcaption{Large Band No Diagonal}\label{fig:two band no diagonal}
    \end{subfigure}
    \begin{subfigure}{0.45\linewidth}
        \includegraphics[width=0.9\linewidth]{Figures/Simulation/Two Small Bands Large Probability Incorrect Equilibrium.png}
        \subcaption{Large Initial Occupation}\label{fig:two band incorrect equilibrium above}
    \end{subfigure}
    \begin{subfigure}{0.45\linewidth}
        \includegraphics[width=0.9\linewidth]{Figures/Simulation/Two Small Bands Small Probability Incorrect Equilibrium.png}
        \subcaption{Small Initial Occupation}\label{fig:two band incorrect equilibrium below}
    \end{subfigure}
    \caption{
        Plots demonstrating the issues
        with the two band approach.
        \cref{fig:two band minimum mixing}
        demonstrates the issue with
        using a large band; the diagonal
        interaction causes mixing within
        groups of eigenstates which
        lifts the degeneracy required for
        the off diagonal interaction.
        This can be fixed by removing
        the diagonal interaction
        (\cref{fig:two band no diagonal})
        however in
        \cref{sec:tunnelling no diagonal}
        we find this is
        necessary for the correct
        tunneling behaviour.
        Even with a small band
        the
        predicted equilibrium
        behaviour is not seen
        (\cref{fig:two band incorrect equilibrium above,fig:two band incorrect equilibrium below}).
        There is therefore very
        little evidence that
        this method
        should be used to model
        the true
        electron-hydrogen dynamics.
    }\label{fig:issue with two band approach}
\end{figure}
Interestingly we do see a reduction in
fluctuations if the system is initially
prepared with this occupation
(\cref{fig:two band close to equilibrium}).
We also find the material cools
as the hydrogen tunnels, as can be seen
through the average electron distribution in
the initial and final state. This is a
result of energy transfer from the
electron gas to the hydrogen
(\cref{fig:two band temperature shift}).

\begin{figure}[htbp]
    \centering
    \begin{subfigure}{0.45\linewidth}
        \includegraphics[width=0.9\linewidth]{Figures/Simulation/Two Small Bands Probability Clsoe To Equilibrium.png}
        \subcaption{Evoulution Close To Equilibrium}\label{fig:two band close to equilibrium}
    \end{subfigure}
    \begin{subfigure}{0.45\linewidth}
        \includegraphics[width=0.9\linewidth]{Figures/Simulation/two band electron distribution.png}
        \subcaption{Two Band Electron Distribution}\label{fig:two band temperature shift}
    \end{subfigure}
    \caption{
        Interesting features of the two
        band dynamics.
        In \cref{fig:two band close to equilibrium}
        we find noise is reduced when the
        hydrogen is prepared in a state
        close to the predicted equilibrium
        occupation. This suggests it is
        true equilibrium of the
        system.
        In \cref{fig:two band temperature shift}
        the temperature shift can be seen.
        This is a consequence of energy transfer
        between the electron gas to the
        hydrogen. The temperature
        shift can be reduced by adding
        more electrons to the
        simulation.
    }\label{fig:final notes two band}
\end{figure}

\section{Simulation Results}\label{sec:simulation results}
% Graphs should show units, and figure captions should explain other variable values such as Temperature, B
% field etc
\subsection{Degenerate tunnelling at 150K}


The curve was initially fitted to
\begin{equation}
    R(N, N) = \frac{R_0}{\cosh{(B(N - 0.5))}}
\end{equation}
with both the initial and final occupancy
equal to N. \ldots however although the rate is small it
would not fall to zero as \(N \rightarrow 0\).
Since we expect the rate to be proportional to the
number of electrons we add a prefactor of \(N(N-1)\)
to the expression, giving us
\begin{equation}
    R = \frac{N(N-1) R_0}{\cosh{(B(N - 0.5))}}
\end{equation}
Although this has \ldots little effect in exp \ldots
when plotting \(\ln{R}\) against \(N\) we can see a
clear reduction in the rate, consistent with this factor
at lower occupations.

TODO-plot, fill out missing \ldots.

The degenerate tunnelling rates

\subsection{Non Degenerate Tunnelling}
To recover the tunnelling rate for hydrogen of different
energy we need to transform the rate equation
into the form

\ldots Discussion of different ways to transform







\section{The Redfield Equation}\label{sec:redfield}

\subsection{General Equation of Motion}
The state of the electron hydrogen
system at a given time can be completely
characterised by its density
matrix.
Working in the interaction
picture, a general density matrix
\(\hat{\rho}(t)\) time evolves according
to the von Neumann equation~\cite{TP2_Notes}.
\begin{equation}
    \frac{d\hat{\rho}_t(t)}{dt} =
    -i [\hat{H}_{int}(t), \hat{\rho}_t(t)]
    \label{eqn:density equation of motion}
\end{equation}
which can be integrated to give
\begin{equation}
    \hat{\rho}_t(t) =
    \hat{\rho}_t(0)
    - i \int_0^t ds
        [\hat{H}_{int}(s), \hat{\rho}_t(s)]
    \label{eqn:integrated density equation of motion}
\end{equation}
We can expand this equation of motion
to second order in the interaction
by substituting \cref{eqn:integrated density equation of motion}
into \cref{eqn:density equation of motion}
twice to give
\begin{equation}
    \frac{d\hat{\rho}_t(t)}{dt} =
    -i [\hat{H}_{int}(t), \hat{\rho}_t(0)]
    - \int_0^t ds
        [\hat{H}_{int}(s),
            [\hat{H}_{int}(s), \hat{\rho}_t(t)]]
    +\mathcal{O}({\hat{H}_{int}}^3)
\end{equation}
It is possible to reduce this to
an equation of motion
describing just the system by taking
a trace over the environment~\cite{Manzano_2020}
\begin{equation}
    \hat{\rho}(t) =
    -i Tr_e[\hat{H}_{int}(t), \hat{\rho}_t(0)]
    - \int_0^t ds
    Tr_e[\hat{H}_{int}(s),
    [\hat{H}_{int}(s), \hat{\rho}_t(t)]]
    \label{eqn:density motion before redfield approximation}
\end{equation}
where \(\hat{\rho}(t) = Tr_e[\hat{\rho}_t(t)]\)
is the density operator of the system.
Using a clever re-definition of the interaction
Hamiltonian~\cite{Manzano_2020} it
is possible to show that the first
term gives no contribution to the
overall dynamics of the system.

\subsection{The Redfield Assumption}\label{sec:the redfield assumption}
To arrive at the Redfield equation
we first make the assumption that the
system and surrounding density
matrix is completely
decoupled~\cite{theory_open_quantum_systems},
allowing us to write
\begin{equation}
    \hat{\rho}_t(t) = \hat{\rho}(t) \otimes \hat{\rho}_E(t)
\end{equation}
where \(\hat{\rho}_E(t)\), the
density matrix of the environment,
is taken as a purely statistical ensemble.
Under the Markov approximation
we can extend the upper limit
of \cref{eqn:density motion before redfield approximation}
to \(\infty \), arriving at the Redfield
equation
\begin{equation}
    \dot{\hat{\rho}}(t) =
    - \int_0^{\infty} ds
    Tr_{E}[\hat{H}_{int}(t),
            [\hat{H}_{int}(s-t),
                    \hat{\rho}(t) \otimes \hat{\rho}_E(t)]]
\end{equation}
Separating out the interaction hamiltonian
into system and surroundings according
to~\cref{eqn:split interaction hamiltonian}
\begin{align}
    \hat{H}_{int} & = \sum_{i,j} \hat{S}_{i,j} \hat{E}_{i,j}
\end{align}
we can simplify the form of this equation to give~\cite{Manzano_2020}
\begin{equation}
    \dot{\hat{\rho{}}}(t) = \begin{aligned}[t]
        \sum_{i,j,k, l} &
        \exp{(-i(\omega_{i,j}-\omega_{k,l})t)}
        \Gamma_{i,j;k, l}(\omega_{k,l})
        [S_{k, l}\hat{\rho}(t),
        S^\dagger_{i,j}]  \\
        +               &
        \exp{(i(\omega_{i,j}-\omega_{k,l}))}
        \Gamma^*_{k, l; i,j}(\omega_{i,j})
        [S_{k, l},
            \hat{\rho}(t) S^\dagger_{i,j}]
    \end{aligned} \label{eqn:redfield equation gamma form}
\end{equation}
where \(\Gamma \) is given by
\begin{equation}
    \Gamma_{i,j, k,l}(\omega) =
    \int_0^\infty{}{
    ds \exp{(i\omega{}s)}
    Tr_{E}[E^\dagger_{i,j}(t)E_{k,l}(t-s)\rho_E(0)]
    }\label{eqn:gamma definition}
\end{equation}

\subsection{The Lindblad Equation}
To obtain the Lindblad equation
we need to apply the rotating
wave approximation to
\cref{eqn:redfield equation gamma form}.
Before applying this
we first expand out the commutators
in \cref{eqn:redfield equation gamma form}
\begin{align}
    \bra{m}[S_{k, l}\hat{\rho}(t),
    S^\dagger_{i, j}] \ket{n}              & =
    \sum_{\alpha, \beta} \rho_{\alpha, \beta} [
        \delta_{m, k}\delta_{l, \alpha}
        \delta_{\beta, j}\delta_{i, n}
        -\delta_{m, j}\delta_{i, k}
    \delta_{l, \alpha}\delta_{\beta, n}]       \\
    \bra{m}[S_{k, l},
    \hat{\rho}(t)S^\dagger_{i, j}] \ket{n} & =
    \sum_{\alpha, \beta} \rho_{\alpha, \beta} [
        \delta_{m, k}\delta_{l, \alpha}
        \delta_{\beta, j}\delta_{i, n}
        -\delta_{m, \alpha}\delta_{\beta, j}
        \delta_{i, k}\delta_{l, n}]
\end{align}
which we use to obtain an expanded
form of \(\gamma{}\)
\begin{equation}
    \bra{m}\dot{\hat{\rho}}(t) \ket{n} = \begin{aligned}[t]
        \sum_{i,j,k, l, \alpha, \beta} &
        \exp{(-i\Delta{}Et)}
        \Gamma_{i,j;k, l}(\omega_{k,l})
        \rho_{\alpha, \beta} [         &
            \delta_{m, k}\delta_{l, \alpha}
        \delta_{\beta, j}\delta_{i, n}                          \\
                                       &                      &
            -\delta_{m, j}\delta_{i, k}
        \delta_{l, \alpha}\delta_{\beta, n}]                    \\
        +                              & \exp{(i\Delta{}E t)}
        \Gamma^*_{k, l; i,j}(\omega_{i,j})
        \rho_{\alpha, \beta} [         &
            \delta_{m, k}\delta_{l, \alpha}
        \delta_{\beta, j}\delta_{i, n}                          \\
                                       &                      &
            - \delta_{m, \alpha}\delta_{\beta, j}
            \delta_{i, k}\delta_{l, n}]
    \end{aligned}
\end{equation}
the rotating wave approximation
limits us to two cases
\begin{itemize}
    \item \(i=j\), \(k=l\)
    \item \(i=k\), \(j=l\)
\end{itemize}
Imposing this condition using the expression
\(\delta_{i,j}\delta_{k,l}
+ \delta_{i,k}\delta_{j,l}
- \delta_{i,j}\delta_{k,l}
\delta_{i,k}\)
we come to the lindblad
equation
\begin{equation}
    \bra{m}\dot{\hat{\rho}}(t) \ket{m}  = \begin{aligned}[t]
        [ & 2\Gamma_{m,\neq m;m, \neq m}(\omega_{m,\neq m})\rho_{\neq m, \neq m} \\
        - & 2\Gamma_{\neq m,m;\neq m, m}(\omega_{\neq m,m})\rho_{m, m}]
    \end{aligned}
\end{equation}
where the off diagonal
terms vanish assuming the
density matrix is initially
purely diagonal. A full derivation
of this expression can be found in
\cref{app:redfield to lindblad}.



\subsection{Calculating \(\Gamma \)}
Using \cref{eqn:gamma definition}
we can calculate the value of \(\Gamma \).
The
density matrix of a purely
statistical ensemble is given
by~\cite{sakurai_napolitano_2020}
\begin{equation}
    \rho_E(0) = \sum_{\{N(k)\}}
    P(\{N(k)\})
    \ket{N(k)} \bra{N(k)}
\end{equation}
and
\begin{equation}
    \hat{E}_{i,j} = \sum_{k,s,k',s'}
    {\tilde{V}_{eff}(\vec{q})}_{i,j}
    \hat{b}^\dagger_{k',s'}\hat{b}_{k,s}
\end{equation}
where we assume the potential
is independent of \(q\). We can
take the trace over the
environment (todo reference)
\begin{equation}
    Tr_E[\dots]  = \begin{aligned}[t]
        \sum_{\substack{\{N(k)\}                             \\
        k_1,s^1,k_2,s^2                                      \\
                k_3,s^3,k_4,s^4 }}
         & P(\{N(k)\}) V_{i,j} V_{k,l}                       \\
         & \exp{(i(E_1 - E_2) t)} \exp{(i(E_3 - E_4) (t-s))} \\
         & \bra{N(k)}
        \hat{b}_{k_1,s^1}^\dagger{} \hat{b}_{k_2,s^2}
        \hat{b}_{k_3,s^3}^\dagger{} \hat{b}_{k_4,s^4}
        \ket{N(k)}
    \end{aligned}
\end{equation}
where \( \{N(k)\} \) is the set of
all possible occupations, and
the boltzmann
probability associated with a
given state is
\(P(\{N(k)\}) = \exp{(-\beta{}(E-\mu N))}\).
The trace is only non zero in two
cases
\begin{itemize}
    \item \(k_1=k_2, s^1=s^2\),
          \(k_3=k_4, s^3=s^4\)
    \item \(k_1=k_4, s^1=s^4\),
          \(k_3=k_2, s^3=s^2\) but
          \(k_1\neq{}k_2, s^1\neq{}s^2\)
\end{itemize}
and we can therefore obtain the
simplified form of the trace (reference)
\begin{equation}
    Tr_E[\dots] = \begin{aligned}[t]
        \sum_{k_1,s^1,k_3,s^3 }
         & V_{i,j} V_{k,l} [ \\
         & N_1 N_3
                + N_1 (1 - N_3) \exp{(-i(E_3 - E_1)s)}]
    \end{aligned}
\end{equation}
Integrating over \(s\) we obtain
an additional constant on the
wavevectors, and after
converting the sum into
an integral and re-adsorbing
the power of \(L^3\) into the
definition of (reference)
\begin{align}
    \Gamma_{i,j, k,l}(\omega) & =\begin{aligned}[t]
        \sum_{s^1,s^3} \int &
        \frac{d^3\vec{k}_1}{{(2\pi)}^3}
        \frac{d^3\vec{k}_3}{{(2\pi)}^3}
        V_{i,j} V_{k,l} [
        N_1 N_3 \delta_{w, 0} \frac{m}{\sqrt{k_3^2}} \\
                            & + N_1 (1 - N_3)
                \frac{m}{\sqrt{k_1^2 - 2m\omega}}
                \delta({k_3 \pm \sqrt{k_1^2 + 2m\omega}}) ]
    \end{aligned}
\end{align}
The first term is divergent (reference)
but we find no terms in the \ldots
with \(\omega = 0\).
We evaluate the second term
by expanding about the fermi
wavevector
\begin{equation}
    \Gamma_{i,j, k,l}(\omega_{k,l}) =\begin{aligned}[t]
        \sum_{s^1,s^3} \exp{(\frac{\beta \omega_{k,l}}{2})} \frac{m k_f^2 }{{(2\pi)}^4}
        V_{i,j} V_{k,l} \sqrt{\pi} \frac{2m}{\beta \hbar^2}
    \end{aligned}
\end{equation}
substituting in the expression
for v (reference)
we arrive at the final
expression for \(\Gamma \)
\begin{equation}
    \sum_{s^1,s^3} \exp{(\frac{\beta \omega_{k,l}}{2})}
    \mathcal{C}_{i,j} \mathcal{C}_{k,l}
    \sqrt{\pi} \frac{8 k_f^2 \epsilon_0^2 \hbar^3}{\beta e^4 m^2}
\end{equation}

\subsection{}
we arrive at the final form of
the Lindblad equation
for our system
\begin{equation}
    \bra{m}\dot{\hat{\rho{}(t)}} \ket{m} = \begin{aligned}[t]
        [ & 2\Gamma_{m,\neq m;m, \neq m}(\omega_{m,\neq m})\rho_{\neq m, \neq m} \\
        - & 2\Gamma_{\neq m,m;\neq m, m}(\omega_{\neq m,m})\rho_{m, m}]
    \end{aligned}
\end{equation}
where
\begin{equation}
    \Gamma_{i,j, k,l}(\omega_{k,l})   =
    \exp{(\frac{\beta \omega_{k,l}}{2})}
    \mathcal{C}_{i,j} \mathcal{C}_{k,l}
    \sqrt{\pi} \frac{32 k_f^2 \epsilon_0^2 \hbar^3}{\beta e^4 m^2}
\end{equation}

\subsection{Analytic Solution to the Rotating Wave Approximation}
Since the form of the rotating
wave approximation is a simple
rate equation with two variables
it is possible to solve it analytically
\cref{app:combined tunnelling rates}.
From the expression above we find the
forward and backward tunnelling rate as
\begin{align}
    \gamma_0 & = 2\Gamma_{1,0;0, 1}(\omega_{1,0})       \\
             & = A \exp{(\frac{\beta \omega_{0,1}}{2})}
    \mathcal{C}_{1,0} \mathcal{C}_{0,1}                 \\
    \gamma_1 & = 2\Gamma_{0,1;1, 0}(\omega_{0,1})       \\
             & = A \exp{(\frac{\beta \omega_{1,0}}{2})} \\
\end{align}
where
\(A =
\mathcal{C}_{1,0} \mathcal{C}_{0,1}
\sqrt{\pi}
\frac{64 k_f^2 \epsilon_0^2 \hbar^3}{\beta e^4 m^2}\).
This gives a combined rate of
\begin{equation}
    \gamma_0 + \gamma_1 = 2A\cosh{(\frac{\beta (E_1 - E_0)}{2})}
    \label{eqn:theoretical rate lindblad equation}
\end{equation}
for an energy difference of
\(3.04\pm0.16\times{}10^{-21} J\)
(\cref{eqn:hydrogen energy difference})
we find a tunnelling rate of
\(6.1\times{}10^{8}s^{-1}\),
corresponding to a
tunnelling time of
\(1.6\times{}10^{-9}s\) at
\(150K\).


\begin{figure}
    \centering
    \includegraphics[width=.5\linewidth]{Figures/Redfield/Plot of lindblad solution.png}
    \caption{Plot of the Lindblad solution with a characteristic decay
    rate of \(6.1\times{}10^{8}s^{-1}\) at
    \(150K\).
    }\label{fig:two site lindblad soluton}
\end{figure}


\subsection{}
In reality there
are 3 HCP sites neighbouring
each FCC Hydrogen, all of which
are connected to 2 other HCP sites.
We should expect the tunnelling
rate to be at least
\(3\) times the single
neighbour rate TODO CITE MODEL CHAPTER s.
To investigate this behaviour we extend
the simulation to contain a large
grid of sites with periodic boundary conditions.
From this we find that the combined FCC occupation
falls at a rate of
\(1.8\times{}10^{9}s^{-1}\) at
\(150K\), exactly
three times the single neighbour rate (\cref{fig:multi site lindblad}).
\begin{figure}[htbp]
    \centering
    \begin{subfigure}{0.45\linewidth}
        \centering
        \includegraphics[width =0.9 \linewidth]{Figures/Redfield/Plot of lindblad solution many sites.png}
        \caption{Individual occupation probability
        }\label{sub@fig:multi site lindblad}
    \end{subfigure}
    \hfill
    \begin{subfigure}{0.45\linewidth}
        \centering
        \includegraphics[width = 0.9\linewidth]{Figures/Redfield/Plot of redfield solution long time.png}
        \caption{Combined occupation probability
        }\label{sub@fig:multi site combined lindblad}
    \end{subfigure}
    \caption{Plot of the individual and combined
    occupation probabilities against time at
    \(150K\). The combined
    probability
    (\cref{sub@fig:multi site combined lindblad})
    follows exactly the same curve as in
    \cref{fig:two site lindblad soluton}
    with a rate of \(1.8\times{}10^{9}s^{-1}\).
    The plot of the individual occupation
    probabilities
    (\cref{sub@fig:multi site lindblad})
    shows that it takes
    a much longer time for the
    probability of occupation of the
    initial site to reach
    an equilibrium with the surroundings.}\label{fig:multi site lindblad}
\end{figure}
It is also possible to infer a
tunneling rate from the
individual probabilities.
Since these decay
exponentially we take the tunneling
time from the time taken for the
distance from equilibrium
to fall by \(\exp{(-1)}\).
For the initial
FCC site we can
take this equilibrium
to be the two site or
the three site occupation.
If we can also
use the time taken for the
neighbouring HCP sites to
reach their maximum occupation,
and half the time taken for the
neighbouring FCC to reach equilibrium.
The corresponding decay times are
given in \cref{tab:implied decay rates}.
\begin{table}[htbp]
    \begin{center}
        \begin{tabular}{ *{3}{c} }
            \toprule
            Measure                & Decay Time \(s\)                  & Implied Decay Rate \(s^{-1}\) \\
            \midrule
            Initial FCC            & \(4.54\times{}10^{-9}\)           & \(2.2\times{}10^{8}\)         \\
            Initial FCC (Two Site) & \(3.99\times{}10^{-10}\)          & \(2.5\times{}10^{9}\)         \\
            Next HCP               & \(4.37\times{}10^{-10}\)          & \(2.3\times{}10^{9}\)         \\
            Next FCC               & \(2\times{}1.31 \times{}10^{-9}\) & \(7.6\times{}10^{8}\)         \\
            Combined Occupation    & \(5.56\times{}10^{-10}\)          & \(1.8\times{}10^{9}\)         \\
            RMS Distance           & \(5.09\times{}10^{-10}\)          & \(2.0 \times 10^{9}\)         \\
            \bottomrule
        \end{tabular}
    \end{center}
    \caption{Implied decay rates
        from the probability
        distribution and RMS distance
        analysis at
        \(150K\).
        Since the individual
        probabilities undergo an exponential
        decay we take the decay time as
        the time for the probability to fall
        by \(\exp{(-1)}\).
        The RMS decay time corresponds
        to the time for the RMS distance
        to equal \(0.5\). Most measures
        of the decay rate take a similar
        amount of time, however the
        complete decay of the FCC
        occupation takes an order of
        magnitude longer.
    }\label{tab:implied decay rates}
\end{table}

\subsection{Distance Traveled}
We should also be able to
extract a tunneling time from
the root-mean squared (rms)
distance travelled by the
hydrogen. As the hydrogen
undergoes a random walk we should
expect the squared distance to
grow linearly with time.
\begin{figure}
    \centering
    \includegraphics[width=0.5\linewidth]{Figures/Redfield/Plot of lindblad solution squared distance.png}
    \caption{Plot of the squared distance
    of the hydrogen atom against time
    showing a linear trend as expected
    for a random walk. The time taken
    for the rms distance to equal \(0.5\)
    is found to be
    \(5.09\times{}10^{-10}s\),
    which corresponds to an implied rate
    of \(2.0 \times 10^{9}s^{-1}\).
    }
\end{figure}
The tunneling time was
found to be \(5.25\times{}10^{-10}s\),
corresponding to teh time taken
for the rms distance to
reach \(0.5\). This gives
an overall rate of
\(1.9 \times 10^{9}s^{-1}\).

\subsection{Comparison with Experiment}
Given these measures of the
tunneling time it is now possible to compare
these tunneling rates to experiment.
In \cref{fig:tunneling rate against temperature}
the theoretical rates
are plotted against temperature
alongside the rates extracted
from the Lindblad analysis.
\begin{figure}[htbp]
    \centering
    \begin{subfigure}{0.45\linewidth}
        \centering
        \includegraphics[width =0.9 \linewidth]{Figures/Redfield/Plot of redfield temperature dependance FCC points.png}
        \caption{FCC Tunneling Rates
        }\label{sub@fig:fcc tunneling rates temp dependence}
    \end{subfigure}
    \hfill
    \begin{subfigure}{0.45\linewidth}
        \centering
        \includegraphics[width = 0.9\linewidth]{Figures/Redfield/Plot of redfield temperature dependance close points.png}
        \caption{Other Tunneling Rates
        }\label{sub@fig:other tunneling rates temp dependence}
    \end{subfigure}
    \caption{Plot of the tunneling
        rates against temperature, with comparison
        to the experimental data.
        The Initial and Next FCC tunneling rates
        (\cref{sub@fig:fcc tunneling rates temp dependence})
        deviate significantly from the experimental
        measurements, as well as many of the other
        measures of tunneling rate.
        The remaining rates however give
        a good fit to the experimental
        data
        (\cref{sub@fig:other tunneling rates temp dependence})
        although the temperature dependence
        is different for different measurements
        of the rate.
    }\label{fig:tunneling rate against temperature}
\end{figure}
Although the FCC tunneling
rates differ by an order of
magnitude the remaining
measurements of the rate
are similar to that measured
in experiment.
TODO-Temp dependance

\subsection{Assessing the Rotating Wave Approximation}
If we relax the rotating wave approximation
we arrive at the expression given in \cref{sec:redfield equation full solution}.
\begin{align}
    \bra{m}\dot{\hat{\rho{}}}(t) \ket{n} & = \begin{aligned}[t]
        \sum_{i,j} &
        \exp{(-i\Delta{}E_{n,j;m,i} t)}
        \Gamma_{n,j;m, i}(\omega_{m,i})
        \rho_{i,j}   \\
                   &
        -\exp{(-i\Delta{}E_{i,m;i,j} t)}
        \Gamma_{i,m;i, j}(\omega_{i,j})
        \rho_{j, n}  \\
                   &
        +\exp{(i\Delta{}E_{n,j;m,i} t)}
        \Gamma_{n,j; m, i}(\omega_{n,j})
        \rho_{i, j}  \\
                   &
        -\exp{(i\Delta{}E_{i,j;i,n} t)}
        \Gamma_{i,j; i, n}(\omega_{i,j})
        \rho_{m, j}
    \end{aligned}
\end{align}
This equation produces extra oscillations
on top of the Lindblad result, with
a characteristic timescales of
\(\frac{2\pi}{\omega_{1,0}} = 2.13\times{}10^{-13}s\). Plotting
the full solution (\cref{fig:redfield full solution})
however we see exactly the same behaviour as that
predicted by the lindblad result.
\begin{figure}[htbp]
    \centering
    \begin{subfigure}{0.45\linewidth}
        \centering
        \includegraphics[width =0.9 \linewidth]{Figures/Redfield/Plot of redfield solution short time.png}
        \caption{Complete solution for small times
        }\label{fig:redfield full solution short timescales}
    \end{subfigure}
    \hfill
    \begin{subfigure}{0.45\linewidth}
        \centering
        \includegraphics[width = 0.9\linewidth]{Figures/Redfield/Plot of redfield solution long time.png}
        \caption{Complete solution for long times
        }\label{fig:redfield full solution long timescales}
    \end{subfigure}
    \begin{subfigure}{0.45\linewidth}
        \centering
        \includegraphics[width = 0.9\linewidth]{Figures/Redfield/Plot of redfield solution long time sink.png}
        \caption{Complete solution with sink
        }\label{fig:redfield full solution with sink}
    \end{subfigure}
    \caption{Plot of the full solution of the Redfield
    equation. On short timescales
    (\cref{fig:redfield full solution short timescales})
    the solution is seen to
    oscillate with a characteristic
    frequency of \(2.1\times{}10^{-13}\)s however
    at long timescales
    (\cref{fig:redfield full solution long timescales})
    the solution decays at the same rate as the
    Lindblad equation. The solution is
    also well behaved with the inclusion of
    a sink at the HCP site
    (\cref{fig:redfield full solution with sink}).
    }\label{fig:redfield full solution}
\end{figure}
In theory we should also be able
to solve the redfield equation
for multiple hydrogen sites, however
the additional computational complexity
rules this out. It is however possible to
approximate the behaviour
seen in the many site model
by placing a sink at
the HCP site
(\cref{fig:redfield full solution with sink}).
In this case
we again see good
agreement with the
lindblad equation.




\section{Improvements and Discussion}\label{sec:improvements}
\subsection{Q dependance}
TODO-
For the purpose of the simulation
it was necessary to assume a q independent
matrix element, \ldots to have a decent number of states \ldots
\ldots half interaction to give effective average

In the future it may also be possible to
directly investigate the effect
of q dependance in a much simpler system
with only one or two separate electron
energy states.
It would then be possible to select
several electron states randomly distributed
around a sphere of k states, and from
this an effective potential could be calculated.


Even in a q-independent model it may
be possible to find an improved effective
average overlap \ldots this could
provide around an additional
factor of 0.5 to the squared overlap

TODO- calculate tunnelling rate if half the
interaction potential.

TODO- Sparse matrix
TODO- Properly vectorise
TODO-

\subsection{Treatment of Electrons}
For the
However the true fermi energy, is slightly lower
at only \(1.24\times{} 10^{-18}J\). This
is possibly due to the two
valence electrons not being fully delocalised.
The true fermi surface
of Nickel is also not completely
spherical~\cite{FermiSufaceNickel},
and has a different level for each spin,
which leads to ferromagnetism~\cite{PhysRev.49.537}.

Since
only a small number of electrons
were included in each simulation
it would not be possible to include
such behaviour, however if we
perform the integration numerically
it should be possible to include these
in the lindblad model.

Ultimately however the effect is small \ldots


\subsection{Multiple Hydrogen Sites}
In the real Nickel lattice the FCC and
HCP sites are arranged in a regular
lattice, such that there are \ldots
neighbouring HCP site of each FCC
hydrogen. In future investigations
it should be therefore be
possible to include the
effect of a larger number of
hydrogen sites,

however
it is unclear how best to
incorporate `hopping' behaviour
of the hydrogen--- in the real
experiment the hydrogen becomes
localised onto a specific site
at later times. An effect we will
fail to see in any \ldots integration of
the schrodinger equation.

\subsection{Generalisation of the Redfield Equation}
One issue with this approach is that it
completely ignores correlations
between the system and the surroundings.
This contradicts
one of the key features seen in the simulation;
the tunnelling process is dominated by
transitions between two states with the same energy,
rather than two states with the same electron configuration.
It is not possible to `trace out' the environment
for an arbitrary coupling, however if
we limit ourselves to
\begin{equation}
    \hat{\rho}_t = \sum_{m,n} \hat{\rho}_{m,n} \otimes {(\hat{\rho}_E)}_{m,n} \
\end{equation}
we can follow the same procedure as in
\cref{sec:the redfield assumption}
to arrive at the modified redfield equation
\begin{equation}
    \bra{m}\dot{\hat{\rho}}(t)\ket{n} = \begin{aligned}[t]
        \sum_{i,j,k, l} &
        \exp{(-i(\omega_{i,j}-\omega_{k,l})t)}
        \Gamma^{m,n}_{i,j;k, l}(\omega_{k,l})
        [S_{k, l}{\hat{\rho}(t)}_{m,n},
        S^\dagger_{i,j}]  \\
        +               &
        \exp{(i(\omega_{i,j}-\omega_{k,l}))}
        {\Gamma^*}^{m,n}_{k, l; i,j}(\omega_{i,j})
        [S_{k, l},
                {\hat{\rho}(t)}_{m,n} S^\dagger_{i,j}]
    \end{aligned}
\end{equation}
where
\begin{equation}
    \Gamma^{m,n}_{i,j, k,l}(\omega) =
    \int_0^\infty{}{
    ds \exp{(i\omega{}s)}
    Tr_{E}[E^\dagger_{i,j}(t)E_{k,l}(t-s)
    {(\hat{\rho}_E)}_{m,n}]
    }
\end{equation}
The problem is then how best to express
both the statistical and quantum uncertainty
in the form of a density matrix.

\subsection{Energy Conservation}
TODO- Reduced Temperature density matrix

TODO- How do we produced thematised and localised states??
we can do one but not both??
TODO- we want states to lie close to the fermi level after
perturbation

% \subsection{Additional Factors}
% \ldots in the derivation we have assumed
% that the \(N \rightarrow N' \)
% tunnelling rate should
% It is however possible that the
% Additional \(N'(N-1)\) factor
% suggested in \cref{eqn:two hop corrected rate}
% suppresses the rate by an extra
% factor of \(\frac{1}{4}\)
% we need more data to be able to say more than
% an order of magnitude approximation

\section{Results}
Graphs should show units, and figure captions should explain other variable values such as Temperature, B
field etc

\section{Discussion}
How do your results fit in with existing knowledge

Improvements
-Electron electron interaction (change \(V(0)\))


\section{Conclusion}\label{sec:conclusion}
What did you discover, what does it imply and what would you do next as a result of your work

\printbibliography{}

\begin{appendix}
    This is the appendix
    \section{Calculating The Potential}\label{app:interaction potential calculation}
    The potential of hydrogen
in the position basis can be found
making use of the greens
function of the coulomb potential\cite{AQP_Problems}
\begin{equation}
    V(\vec{r}) = \frac{e^2}{4 \pi \epsilon_0}(
    -\frac{1}{r}
    + \int{\frac{\abs{\psi(\vec{r})}^2}{
            \abs{\vec{r} - \vec{r'}}} d\vec{r}'})
\end{equation}
when we assume the electron
lies in the 1s state\cite{griffiths_schroeter_2018}
\begin{equation}
    \psi(\vec{r}) = {(\pi a_0^3)}^{\frac{1}{2}} e^{-\frac{r}{a_0}}
\end{equation}
we arrive at the following potential
\begin{equation}
    V(\vec{r}) = \frac{e^2}{4 \pi \epsilon_0}(
    -\frac{1}{r}
    + \pi a_0^3\int{\frac{e^{-\frac{2r}{a_0}}}{
            \abs{\vec{r} - \vec{r'}}} d\vec{r}'})
\end{equation}
If we make use of two standard results
\begin{align}
    \int{\frac{1}{r} e^{i\vec{q}.\vec{r}} d^3\vec{r}}
     & = \frac{4 \pi}{q^2}                         \\
    \int{e^{-\alpha r} e^{i\vec{q}.\vec{r}} d^3\vec{r}}
     & = \frac{8 \pi \alpha}{{(\alpha^2 + q^2)}^2} \\
\end{align}
with \(\alpha = \frac{2}{a_0}\) we end up
with
\begin{align}
    V(\vec{q}) & = \frac{e^2}{4 \pi \epsilon_0}(
    - \frac{4\pi}{q^2}
    + \frac{\pi a_0^3}{q^2}
    \frac{8 \pi \alpha}{{(\alpha^2 + q^2)}^2})   \\
               & = \frac{e^2}{\epsilon_0 q^2}(
    \frac{ \alpha^4}{{(\alpha^2 + q^2)}^2} - 1
    )                                            \\
               & =-\frac{e^2}{\epsilon_0 }
    \frac{ 2\alpha^2 + q^2 }{{(\alpha^2 + q^2)}^2}
\end{align}

    \section{Combined Tunnelling Rates}\label{app:combined tunnelling rates}
    
Suppose we have a system with two states
\(x_1, x_2\)
with a forward and backwards
tunnelling rate \(\gamma_1, \gamma_2\).
We can express the equations of motion
in matrix form
\begin{align}
    \begin{pmatrix}
        \dot{x}_1 \\
        \dot{x}_2
    \end{pmatrix} = \begin{pmatrix}
        -\gamma_1 & \gamma_2   \\
        \gamma_1  & - \gamma_2
    \end{pmatrix}\begin{pmatrix}
        x_1 \\
        x_2
    \end{pmatrix}
\end{align}
from here we find the eigenvalues
and eigenvectors of the matrix.
\begin{alignat}{3}
    \lambda_1  = & -\gamma_1 - \gamma_2
                 & \Rightarrow V_1  = a\begin{pmatrix}
        1 \\
        -1
    \end{pmatrix} \\
    \lambda_2  = & 0
                 & \Rightarrow V_1  =a\begin{pmatrix}
        \frac{\gamma_2}{\gamma_1} \\
        1
    \end{pmatrix}
\end{alignat}

Given a state initially in 1
we can decompose it into these
eigenstates to find the behaviour
at a later time

\begin{alignat}{2}
    \begin{pmatrix}
        x_1(0) \\
        x_2(0)
    \end{pmatrix}              & =\begin{pmatrix}
        1 \\
        0
    \end{pmatrix} \\
                                           & =
    \frac{1}{1 + \frac{\gamma_2}{\gamma_1}}(\begin{pmatrix}
        \frac{\gamma_2}{\gamma_1} \\
        1
    \end{pmatrix} +
    \begin{pmatrix}
        1 \\
        -1
    \end{pmatrix})                                          \\
    \Rightarrow \begin{pmatrix}
        x_1(t) \\
        x_2(t)
    \end{pmatrix} & =
    \frac{1}{1 + \frac{\gamma_2}{\gamma_1}}(\begin{pmatrix}
        \frac{\gamma_2}{\gamma_1} \\
        1
    \end{pmatrix} +
    \begin{pmatrix}
        1 \\
        -1
    \end{pmatrix}\exp{(-(\gamma_1 + \gamma_2)t)})
\end{alignat}
The system therefore decays at a rate
proportional to \(\gamma_1 + \gamma_2\)
and reaches an equilibrium at
\begin{equation}
    \begin{pmatrix}
        x_1(\infty) \\
        x_2(\infty)
    \end{pmatrix}  =\frac{1}{\gamma_1 + \gamma_2} \begin{pmatrix}
        \gamma_2 \\
        \gamma_1
    \end{pmatrix}
\end{equation}


    \section{From Redfield to the Lindblad Equation}\label{app:redfield to lindblad}
    Starting from the
expression for \(\dot{\rho}(t)\)
in \cref{sec:the redfield assumption}
\begin{align}
    \dot{\hat{\rho}}(t) = \begin{aligned}[t]
        \sum_{i,j,k, l} &
        \exp{(-i(\omega_{i,j}-\omega_{k,l})t)}
        \Gamma_{i,j;k, l}(\omega_{k,l})
        [S_{k, l}\hat{\rho}(t),
        S^\dagger_{i,j}]                                       \\
        +               & \exp{(i(\omega_{i,j}-\omega_{k,l}))}
        \Gamma^*_{k, l; i,j}(\omega_{i,j})
        [S_{k, l},
            \hat{\rho}(t) S^\dagger_{i,j}]
    \end{aligned}
\end{align}
where \(S_{i,j}= \hat{a}^\dagger_i \hat{a}_j\),
\(\omega_{i,j} = E_i - E_j\), the energy of
the hydrogen atoms excluding the interaction, and
gamma is as defined in
\cref{eqn:gamma definition}. To
make the following calculation easier we deal
with a single component of \(\rho \), and
denote \(\Delta{}E = \omega_{i,j}-\omega_{k,l}\)
\begin{align}
    \bra{m}\dot{\hat{\rho}}(t) \ket{n} = \begin{aligned}[t]
        \sum_{i,j,k, l} &
        \exp{(-i\Delta{}Et)}
        \Gamma_{i,j;k, l}(\omega_{k,l})
        \bra{m}[S_{k, l}\hat{\rho}(t),
        S^\dagger_{i,j}] \ket{n}               \\
        +               & \exp{(i\Delta{}E t)}
        \Gamma^*_{k, l; i,j}(\omega_{i,j})
        \bra{m}[S_{k, l},
            \hat{\rho}(t) S^\dagger_{i,j}]\ket{n}
    \end{aligned}
\end{align}
focusing on the commutators we have
\begin{align}
    \bra{m}[S_{k, l}\hat{\rho}(t),
    S^\dagger_{i, j}] \ket{n} & =
    \sum_{\alpha, \beta} \rho_{\alpha, \beta}\bra{m}
    [\hat{a}^\dagger_k \hat{a}_l
        \hat{a}^\dagger_\alpha \hat{a}_\beta,
        \hat{a}^\dagger_j \hat{a}_i]
    \ket{n}                       \\
                              & =
    \sum_{\alpha, \beta} \rho_{\alpha, \beta}\bra{m}
    \hat{a}^\dagger_k \hat{a}_l
    \hat{a}^\dagger_\alpha \hat{a}_\beta
    \hat{a}^\dagger_j \hat{a}_i
    -
    \hat{a}^\dagger_j \hat{a}_i
    \hat{a}^\dagger_k \hat{a}_l
    \hat{a}^\dagger_\alpha \hat{a}_\beta
    \ket{n}                       \\
                              & =
    \sum_{\alpha, \beta} \rho_{\alpha, \beta} [
        \delta_{m, k}\delta_{l, \alpha}
        \delta_{\beta, j}\delta_{i, n}
        -\delta_{m, j}\delta_{i, k}
        \delta_{l, \alpha}\delta_{\beta, n}]
\end{align}
similarly for the second commutator
\begin{align}
    \bra{m}[S_{k, l},
    \hat{\rho}(t)S^\dagger_{i, j}] \ket{n} & =
    \sum_{\alpha, \beta} \rho_{\alpha, \beta}\bra{m}
    [\hat{a}^\dagger_k \hat{a}_l
        ,\hat{a}^\dagger_\alpha \hat{a}_\beta
        \hat{a}^\dagger_j \hat{a}_i]
    \ket{n}                                    \\
                                           & =
    \sum_{\alpha, \beta} \rho_{\alpha, \beta} [
        \delta_{m, k}\delta_{l, \alpha}
        \delta_{\beta, j}\delta_{i, n}
        -\delta_{m, \alpha}\delta_{\beta, j}
        \delta_{i, k}\delta_{l, n}]
\end{align}
subbing into the full expression for
\(\Gamma \) we find
\begin{align}
    \bra{m}\dot{\hat{\rho}}(t) \ket{n} = \begin{aligned}[t]
        \sum_{i,j,k, l, \alpha, \beta} &
        \exp{(-i\Delta{}Et)}
        \Gamma_{i,j;k, l}(\omega_{k,l})
        \rho_{\alpha, \beta} [         &
            \delta_{m, k}\delta_{l, \alpha}
        \delta_{\beta, j}\delta_{i, n}                          \\
                                       &                      &
            -\delta_{m, j}\delta_{i, k}
        \delta_{l, \alpha}\delta_{\beta, n}]                    \\
        +                              & \exp{(i\Delta{}E t)}
        \Gamma^*_{k, l; i,j}(\omega_{i,j})
        \rho_{\alpha, \beta} [         &
            \delta_{m, k}\delta_{l, \alpha}
        \delta_{\beta, j}\delta_{i, n}                          \\
                                       &                      &
            - \delta_{m, \alpha}\delta_{\beta, j}
            \delta_{i, k}\delta_{l, n}]
    \end{aligned}
\end{align}

\subsection{Rotating Wave Approximation}
To apply the rotating wave approximation we consider
two cases
\begin{itemize}
    \item \(i=j\), \(k=l\)
    \item \(i=k\), \(j=l\)
\end{itemize}
however we need to make sure we don't
double count. This is done using
the following delta
function
\(\delta_{i,j}\delta_{k,l}
+ \delta_{i,k}\delta_{j,l}
- \delta_{i,j}\delta_{k,l}
\delta_{i,k}\)
this is done on each group of
\(\delta \) functions separately
\begin{align}
    (1) & =  (\delta_{i,j}\delta_{k,l}
    + \delta_{i,k}\delta_{j,l}
    - \delta_{i,j}\delta_{k,l}\delta_{i,k}) (
    \delta_{m, k}\delta_{l, \alpha}
    \delta_{\beta, j}\delta_{i, n})                          \\
        & =  (\delta_{m, k, l, \alpha}\delta_{\beta, j,i, n}
    + \delta_{i, n, m, k}\delta_{l, \alpha, \beta, j}
    - \delta_{m, k, l, \alpha, \beta, j,i, n})               \\
    (2) & =  (\delta_{i,j}\delta_{k,l}
    + \delta_{i,k}\delta_{j,l}
    - \delta_{i,j}\delta_{k,l}
    \delta_{i,k}) (
    \delta_{m, j}\delta_{i, k}
    \delta_{l, \alpha}\delta_{\beta, n})                     \\
        & =  (\delta_{m, j,i, k,l, \alpha}\delta_{\beta, n}
    + \delta_{m, j,l, \alpha}\delta_{i, k}\delta_{\beta, n}
    - \delta_{m, j,i, k, l, \alpha}\delta_{\beta, n} )       \\
    (3) & =  (\delta_{i,j}\delta_{k,l}
    + \delta_{i,k}\delta_{j,l}
    - \delta_{i,j}\delta_{k,l}
    \delta_{i,k}) (
    \delta_{m, \alpha}\delta_{\beta, j}
    \delta_{i, k}\delta_{l, n})                              \\
        & =  (\delta_{m, \alpha}\delta_{\beta, j,i, k, l, n}
    + \delta_{m, \alpha}\delta_{\beta, j,l, n}
    \delta_{i, k}
    - \delta_{m, \alpha}\delta_{\beta, j,i, k,l, n})
\end{align}
applying this to the full expression
for gamma we find



\begin{align}
    \bra{m}\dot{\hat{\rho}}(t) \ket{n} & = \begin{aligned}[t]
        \sum_{i,j,k, l, \alpha, \beta} &
        \Gamma_{i,j;k, l}(\omega_{k,l})
        \rho_{\alpha, \beta} [                             \\
                                       &
            (\delta_{m, k, l, \alpha}\delta_{\beta, j,i, n}
            + \delta_{i, n, m, k}\delta_{l, \alpha, \beta, j}
        - \delta_{m, k, l, \alpha, \beta, j,i, n})         \\
                                       &
            - (    \delta_{m, j,i, k,l, \alpha}\delta_{\beta, n}
            + \delta_{m, j,l, \alpha}\delta_{i, k}\delta_{\beta, n}
        - \delta_{m, j,i, k, l, \alpha}\delta_{\beta, n})] \\
        +                              &
        \Gamma^*_{k, l; i,j}(\omega_{i,j})
        \rho_{\alpha, \beta} [                             \\
                                       &
            (\delta_{m, k, l, \alpha}\delta_{\beta, j,i, n}
            + \delta_{i, n, m, k}\delta_{l, \alpha, \beta, j}
        - \delta_{m, k, l, \alpha, \beta, j,i, n})         \\
                                       &
            - (\delta_{m, \alpha}\delta_{\beta, j,i, k, l, n}
            + \delta_{m, \alpha}\delta_{\beta, j,l, n}
            \delta_{i, k}
            - \delta_{m, \alpha}\delta_{\beta, j,i, k,l, n})]
    \end{aligned}  \\
                                       & = \begin{aligned}[t]
        \sum_{i} &
        [ (\Gamma_{n,n;m, m}(\omega_{m,m})\rho_{m, n}\delta_{i,n}
        + \Gamma_{n,i;n, i}(\omega_{n,i})\rho_{i, i}\delta_{m,n}     \\ &
        - \Gamma_{n,n;n,n}(\omega_{n,n})\rho_{n,n}\delta_{m,n,i})    \\
                 &
                - (\Gamma_{m,m;m,m}(\omega_{m,m})\rho_{m, n}\delta_{i,n}
        +\Gamma_{i,m;i, m}(\omega_{i,m})\rho_{m, n}                  \\ &
        -\Gamma_{m,m;m,m}(\omega_{m,m})\rho_{m, n}\delta_{i,n})]     \\
        +        &
        [(\Gamma^*_{m,m; n,n}(\omega_{n,n})\rho_{m, n}\delta_{i,n}
        + \Gamma^*_{m,i; m,i}(\omega_{m,i})\rho_{i, i}\delta_{n, m}  \\ &
        - \Gamma^*_{n,n,n,n}(\omega_{n,n})\rho_{n,n}\delta_{n, m,i}) \\
                 &
                - (\Gamma^*_{n,n;n,n}(\omega_{n,n})\rho_{m, n}\delta_{i,n}
        + \Gamma^*_{i, n; i,n}(\omega_{i,n})\rho_{m, n}              \\ &
                - \Gamma^*_{n,n,n,n}(\omega_{n,n})\rho_{m, n}\delta_{i, n})]
    \end{aligned}  \\
                                       & = \begin{aligned}[t]
        \sum_{i} &
        [ \Gamma_{n,n;m, m}(\omega_{m,m})\rho_{m, n}\delta_{i,n}
        + \Gamma_{n,i;n, i}(\omega_{n,i})\rho_{i, i}\delta_{m,n}    \\
                 &
                - \Gamma_{n,n;n,n}(\omega_{n,n})\rho_{n,n}\delta_{m,n,i}
        - \Gamma_{i,m;i, m}(\omega_{i,m})\rho_{m, n}]               \\
        +        &
        [ \Gamma^*_{m,m; n,n}(\omega_{n,n})\rho_{m, n}\delta_{i,n}
        + \Gamma^*_{m,i; m,i}(\omega_{m,i})\rho_{i, i}\delta_{n, m} \\
                 &
                - \Gamma^*_{n,n,n,n}(\omega_{n,n})\rho_{n,n}\delta_{n, m,i}
                - \Gamma^*_{i,n; i,n}(\omega_{i,n})\rho_{m, n}]
    \end{aligned}
\end{align}
to convert \(\Gamma^*\)
into \(\Gamma \) we use
\(\Gamma^*_{a,b,c,d} = \Gamma_{c, d, a, b}\)
\begin{align}
    \bra{m}\dot{\hat{\rho}}(t) \ket{n} & = \begin{aligned}[t]
        \sum_{i} &
        [ \Gamma_{n,n;m, m}(\omega_{m,m})\rho_{m, n}\delta_{i,n}
        + \Gamma_{n,i;n, i}(\omega_{n,i})\rho_{i, i}\delta_{m,n} \\
                 &
                - \Gamma_{n,n;n,n}(\omega_{n,n})\rho_{n,n}\delta_{m,n,i}
        - \Gamma_{i,m;i, m}(\omega_{i,m})\rho_{m, n}]            \\
        +        &
        [ \Gamma_{n,n; m, m}(\omega_{m,m})\rho_{m, n}\delta_{i,n}
        + \Gamma_{m,i;m,i}(\omega_{m,i})\rho_{i, i}\delta_{n, m} \\
                 &
                - \Gamma_{n,n,n,n}(\omega_{n,n})\rho_{n,n}\delta_{n, m,i}
                - \Gamma_{ i,n; i,n}(\omega_{n,i})\rho_{m, n}]
    \end{aligned} \\
                                       & = \begin{aligned}[t]
        \sum_{i} &
        [2 \Gamma_{n,n;m, m}(\omega_{m,m})\rho_{m, n}\delta_{i,n}
        + 2\Gamma_{n,i;n, i}(\omega_{n,i})\rho_{i, i}\delta_{m,n} \\
                 &
                - 2\Gamma_{n,n;n,n}(\omega_{n,n})\rho_{n,n}\delta_{m,n,i}
        - \Gamma_{i,m;i, m}(\omega_{i,m})\rho_{m, n}              \\
                 &
                - \Gamma_{ i,n; i,n}(\omega_{n,i})\rho_{m, n}]
    \end{aligned}
\end{align}
the second and third terms cancel
for \(i=m\), and the first and
last two terms cancel for \(m=n\)
provided \(i = n\)
\begin{align}
    \bra{m}\dot{\hat{\rho}}(t) \ket{n} & = \begin{aligned}[t]
        \sum_{i} &
        [2 \Gamma_{n,n;m, m}(\omega_{m,m})\rho_{m, n}\delta_{i,n}
        + 2\Gamma_{n,\neq n;n, \neq n}(\omega_{n,\neq n})\rho_{\neq n, \neq n}\delta_{m,n, i} \\
                 &
                - \Gamma_{i,m;i, m}(\omega_{i,m})\rho_{m, n}
                - \Gamma_{ i,n;i,n}(\omega_{n,i})\rho_{m, n}]
    \end{aligned}
\end{align}
If we take \(m=n\)
\begin{align}
    \bra{m}\dot{\hat{\rho}}(t) \ket{m} & = \begin{aligned}[t]
        \sum_{i, n\neq m} &
        [2 \Gamma_{m,m;m, m}(\omega_{m,m})\rho_{m, m}\delta_{i,m} \\ &
        + 2\Gamma_{m,n;m, n}(\omega_{m,n})\rho_{n, n}\delta_{m,i} \\
                          &
                - 2 \Gamma_{i,m;i, m}(\omega_{i,m})\rho_{m, m}]
    \end{aligned} \\
                                       & = \begin{aligned}[t]
        2\sum_{i,  n\neq m} &
        [\Gamma_{m,n;m, n}(\omega_{m,n})\rho_{n, n}\delta_{m,i} \\
                            &
                - \Gamma_{n,m;n, m}(\omega_{n,m})\rho_{m, m}\delta_{m,i}]
    \end{aligned} \\
                                       & =
    2\sum_{n\neq m}[
        \Gamma_{m,n;m, n}(\omega_{m,n})\rho_{n, n}
        - \Gamma_{n,m;n, m}(\omega_{n,m})\rho_{m, m}]
    \label{eqn:cross terms density matrix evolution}
\end{align}
If we take \(m \neq n\)
\begin{align}
    \bra{m}\dot{\hat{\rho}}(t) \ket{\neq m} & = \begin{aligned}[t]
        \sum_{i} &
        [2 \Gamma_{\neq m,\neq m;m, m}(\omega_{m,m})\rho_{m, \neq m}\delta_{i,m} \\
                 &
                - \Gamma_{i,m;i, m}(\omega_{i,m})\rho_{m, \neq m}
                - \Gamma_{ i,\neq m; i, \neq m}(\omega_{i, \neq m,})\rho_{m, \neq m}]
    \end{aligned}
\end{align}
We only have self interaction
in the diagonal elements.
If they start 0 they will
always remain zero at
later times. If we
sum over \(m\) for the
case \(n=m\) we find
the total derivative is
zero, the normalisation of
the density matrix is preserved.


\subsection{Full Solution}\label{sec:redfield equation full solution}
The full solution to the equation is
much more complicated. Starting from the
previous expression for \(\rho \) we
swap \(\Gamma^*\) for \(\Gamma \) and expand
\begin{align}
    \bra{m}\dot{\hat{\rho}}(t) \ket{n} & = \begin{aligned}[t]
        \sum_{i,j,k, l, \alpha, \beta} &
        \exp{(-i\Delta{}E_{i,j;k,l}t)}
        \Gamma_{i,j;k, l}(\omega_{k,l})
        \rho_{\alpha, \beta} [               \\ &
            \delta_{m, k}\delta_{l, \alpha}
            \delta_{\beta, j}\delta_{i, n}
            -\delta_{m, j}\delta_{i, k}
        \delta_{l, \alpha}\delta_{\beta, n}] \\
        +                              &
        \exp{(i\Delta{}E_{i,j;k,l} t)}
        \Gamma_{i,j; k, l}(\omega_{i,j})
        \rho_{\alpha, \beta} [               \\ &
            \delta_{m, k}\delta_{l, \alpha}
            \delta_{\beta, j}\delta_{i, n}
            - \delta_{m, \alpha}\delta_{\beta, j}
            \delta_{i, k}\delta_{l, n}]
    \end{aligned}                                   \\
                                       & = \begin{aligned}[t]
        \sum_{i,j} &
        \exp{(-i\Delta{}E_{n,j;m,i} t)}
        \Gamma_{n,j;m, i}(\omega_{m,i})
        \rho_{i,j}   \\
                   &
        -\exp{(-i\Delta{}E_{i,m;i,j} t)}
        \Gamma_{i,m;i, j}(\omega_{i,j})
        \rho_{j, n}  \\
                   &
        +\exp{(i\Delta{}E_{n,j;m,i} t)}
        \Gamma_{n,j; m, i}(\omega_{n,j})
        \rho_{i, j}  \\
                   &
        -\exp{(i\Delta{}E_{i,j;i,n} t)}
        \Gamma_{i,j; i, n}(\omega_{i,j})
        \rho_{m, j}
    \end{aligned}\label{eqn:full redfield solution}
\end{align}
consider the case \(m=n\)
\begin{align}
    \bra{m}\dot{\hat{\rho}}(t) \ket{m} & = \begin{aligned}[t]
        \sum_{i,j} &
        \exp{(-i\omega_{i,j} t)}
        \Gamma_{m,j;m, i}(\omega_{m,i})
        \rho_{i,j}   \\
                   &
        -\exp{(-i\omega_{j,m} t)}
        \Gamma_{i,m;i, j}(\omega_{i,j})
        \rho_{j, m}  \\
                   &
        +\exp{(i\omega_{i,j} t)}
        \Gamma_{m,j; m, i}(\omega_{m,j})
        \rho_{i, j}  \\
                   &
        -\exp{(i\omega_{m,j} t)}
        \Gamma_{i,j; i, m}(\omega_{i,j})
        \rho_{m, j}
    \end{aligned} \\
                                       & = \begin{aligned}[t]
        \sum_{i,j} &
        (\exp{(-i\omega_{i,j} t)}
        \Gamma_{m,j;m, i}(\omega_{m,i})
        +\exp{(i\omega_{i,j} t)}
        \Gamma_{m,j; m, i}(\omega_{m,j})
        )\rho_{i,j}                 \\
                   &
        -\exp{(-i\omega_{j,m} t)}
        \Gamma_{i,m;i, j}(\omega_{i,j})
        (\rho_{j, m} + \rho_{m, j}) \\
    \end{aligned}
\end{align}
we need to make sure that this
has no contribution for
\(\Gamma(0)\), as this is divergent
(see \cref{eqn:divergent expression for first integral}).
We only need
to worry about terms not already
covered by the rotating wave approximation.
This covers only one case for each part of
the expression
\begin{itemize}
    \item \(i \neq j\), \(i \neq m\) in the
          first expression
    \item \(j\neq m\), \(i \neq j\) on the
          second ect \ldots
\end{itemize}
the contribution from these
terms is then
\begin{align}
    \begin{aligned}
         & \exp{(-i\omega_{\neq m,m} t)}
        \Gamma_{m,m;m, \neq m}(\omega_{m,\neq m})
        \rho_{\neq m,m}                  \\
         &
        +\exp{(i\omega_{m,\neq m} t)}
        \Gamma_{m,\neq m; m, m}(\omega_{m,\neq m})
        \rho_{m,\neq m}                  \\
         &
        -\exp{(-i\omega_{\neq m,m} t)}
        \Gamma_{m,m;m, \neq m}(\omega_{m,\neq m})
        (\rho_{\neq m, m} + \rho_{m, \neq m})
    \end{aligned}
\end{align}
It is clear that these terms all cancel,
and the divergence of \(\Gamma \) at \(\omega = 0\)
has no effect on the diagonal terms of the
density matrix. By a similar process the same can
be shown for cross diagonal terms.

    \section{Calculating Gamma For The Lindblad Equation}
    Starting from the definition of
the Lindblad rate constant \(\gamma{}\)
(\cref{eqn:gamma definition}) and
the environment interaction hamiltonian
(\cref{eqn:split interaction hamiltonian})
we find
\begin{equation}
    \Gamma_{i,j, k,l}(\omega) =
    \int_0^\infty{}{
    ds \exp{(i\omega{}s)}
    Tr_{E}[E^\dagger_{i,j}(t)E_{k,l}(t-s)\rho_E(0)]
    }
\end{equation}
where
\begin{align}
    E_{i, j}(t) & =
    \exp{(iH_e t)}
    \sum_{k,k'} V_{i,j} \hat{b}^\dagger_{k',s'}\hat{b}_{k,s}
    \exp{(-iH_e t)}                           \\
                & = \sum_{k,k'} \hat{V}_{i,j}
    \hat{b}^\dagger_{k',s'}\hat{b}_{k,s} \exp{(i(E_k' - E_k)t)}
\end{align}
For a purely
statistical ensemble of electrons
the density matrix is
diagonal~\cite{sakurai_napolitano_2020}
\begin{equation}
    \rho_E(0) = \sum_{\{N(k)\}}
    P(\{N(k)\})
    \ket{N(k)} \bra{N(k)}
\end{equation}
where \(P(N(k)) =
\frac{1}{z}\sum_{k,s}
\exp{(-{N(k)}_s(\beta E_k - \mu))}\).

We start by expanding out the trace
over the environment
\begin{align}
    Tr_E[\dots] & = \sum_{\{N(k)\}}
    \bra{N(k)} E^\dagger_{i,j}(t)E_{k,l}(t-s) \rho_E(0) \ket{N(k)}
    \\
                & = \begin{aligned}[t]
        \sum_{\{N(k)\}, \{N'(k)\}} &
        P(\{N'(k)\}) \bra{N'(k)}  \ket{N(k)}                                               \\
                                   & \bra{N(k)} E^\dagger_{i,j}(t)E_{k,l}(t-s) \ket{N'(k)}
    \end{aligned} \\
                & = \sum_{\{N(k)\}}
    P(\{N(k)\}) \bra{N(k)}
    E^\dagger_{i,j}(t)E_{k,l}(t-s) \ket{N(k)} \\
                & = \begin{aligned}[t]
        \sum_{\substack{\{N(k)\}                             \\
        k_1,s^1,k_2,s^2                                      \\
                k_3,s^3,k_4,s^4 }}
         & P(\{N(k)\}) V_{i,j} V_{k,l}                       \\
         & \exp{(i(E_1 - E_2) t)} \exp{(i(E_3 - E_4) (t-s))} \\
         & \bra{N(k)}
        \hat{b}_{k_1,s^1}^\dagger{} \hat{b}_{k_2,s^2}
        \hat{b}_{k_3,s^3}^\dagger{} \hat{b}_{k_4,s^4}
        \ket{N(k)}
    \end{aligned}
\end{align}
Where here we have used
the fact that the
potential is real, and
swapped the order of \(k_1, k_2\)
from the usual definition.
This expression is non zero
in only two cases
\begin{itemize}
    \item \(k_1=k_2, s^1=s^2\),
          \(k_3=k_4, s^3=s^4\)
    \item \(k_1=k_4, s^1=s^4\),
          \(k_3=k_2, s^3=s^2\) but
          \(k_1\neq{}k_2, s^1\neq{}s^2\)
\end{itemize}
we use the result
\begin{align}
    \bra{N(k)}
    \hat{b}_{k_1,s^1}^\dagger{}
    \hat{b}_{k_1,s^1}
    \hat{b}_{k_3,s^3}^\dagger{}
    \hat{b}_{k_3,s^3}\ket{N(k)} & = N_1 N_3       \\
    \bra{N(k)}
    \hat{b}_{k_1,s^1}^\dagger{}
    \hat{b}_{k_3,s^3}
    \hat{b}_{k_3,s^3}^\dagger{}
    \hat{b}_{k_1,s^1}\ket{N(k)} & = \bra{N(k)}
    \hat{b}_{k_1,s^1}^\dagger{}
    \hat{b}_{k_1,s^1}
    \hat{b}_{k_3,s^3}
    \hat{b}_{k_3,s^3}^\dagger{}
    \ket{N(k)}                                    \\
                                & = N_1 (1 - N_3)
\end{align}
to simplify the above sum
\begin{equation}
    Tr_E[\dots] = \begin{aligned}[t]
        \sum_{k_1,s^1,k_3,s^3 }
         & V_{i,j} V_{k,l} [ \\
         & N_1 N_3
                + N_1 (1 - N_3) \exp{(-i(E_3 - E_1)s)}]
    \end{aligned}
\end{equation}
If we then integrate over s we find
\begin{align}
    \Gamma_{i,j, k,l}(\omega) & =
    \int_0^\infty{}{
        ds \exp{(i\omega{}s)} Tr_{E}[\dots]
    }                                                      \\
    {}                        & =\begin{aligned}[t]
        \sum_{k_1,s^1,k_3,s^3,k_4,s^4 }
         & V_{i,j} V_{k,l} [ \\
         & N_1 N_3 \delta(w)
                + N_1 (1 - N_3)  \delta(w + E_1 -E_3) ]
    \end{aligned}
\end{align}
to convert these delta functions
into delta function in momentum we
use the formula
\(\delta(f(x)) =
\abs{\frac{df}{dx}}^{-1}\delta(x - x_0)\)
to give
\begin{equation}
    \delta(w + E_1 -E_3) =
    \frac{m_e}{\sqrt{k_1^2 - 2m_e\omega}}
    \delta({k_3 \pm \sqrt{k_1^2 + 2m_e\omega}})
\end{equation}
Note we are working in units of \(\hbar = 1\)
and the \(\delta(\omega)\) is just
constraining \(E_3 = E_4\) which is
already satisfied when \(k_3 = k_4\).
Adding this back into the expression
for \(\Gamma \) we find
\begin{align}
    \Gamma_{i,j, k,l}(\omega) & =\begin{aligned}[t]
        \sum_{k_1,s^1,k_3,s^3 }
         & V_{i,j} V_{k,l} [
        N_1 N_3 \delta_{w, 0} \frac{m_e}{\sqrt{k_3^2}} \\
         & + N_1 (1 - N_3)
                \frac{m_e}{\sqrt{k_1^2 - 2m_e\omega}}
                \delta({k_3 \pm \sqrt{k_1^2 + 2m_e\omega}}) ]
    \end{aligned}
\end{align}
To calculate these terms we
need to switch to the integral
representation. Absorbing the factors
of \(L^6\) back into the definition
of \(V_{i,j}\) we find
\begin{align}
    \Gamma_{i,j, k,l}(\omega) & =\begin{aligned}[t]
        \sum_{s^1,s^3} \int &
        \frac{d^3\vec{k}_1}{{(2\pi)}^3}
        \frac{d^3\vec{k}_3}{{(2\pi)}^3}
        V_{i,j} V_{k,l} [
        N_1 N_3 \delta_{w, 0} \frac{m_e}{\sqrt{k_3^2}} \\
                            & + N_1 (1 - N_3)
                \frac{m_e}{\sqrt{k_1^2 - 2m_e\omega}}
                \delta({k_3 \pm \sqrt{k_1^2 + 2m_e\omega}}) ]
    \end{aligned}
\end{align}
we perform the integral over k by expanding about
\(k = k_f\) noting that the value of
\(\omega \) is equal to the energy
difference of the hydrogen
\(\Gamma_{i,j, k,l}(\omega) = \Gamma_{i,j, k,l}(\omega_{k,l})\)
where \(\omega_{k,l} = E_k - E_l\)
(see \cref{eqn:cross terms density matrix evolution}).
\begin{align}
    1 - N_3 & = \frac{1}{1 + \exp{(-\beta(E_3 - \mu))}}                                  \\
            & = \frac{1}{1 + \exp{(-\beta(E_1 + \omega - \mu))}}                         \\
            & \sim \frac{1}{2 + -\beta(E_1 + \omega - \mu)}                              \\
            & \sim \frac{1}{(1 - \frac{\beta \omega}{2})(2 + -\beta(E_1  - \mu))}        \\
            & \sim \exp{(\frac{\beta \omega}{2})}\frac{1}{1 + \exp{(-\beta(E_1 - \mu))}}
\end{align}

We then discard the first part of the integral by noting no
terms with \(\omega = 0\) appear in the
final expression for \(\dot{\rho}\), and note
\(\omega \ll k_1\) at the fermi surface.
\begin{align}
    \Gamma_{i,j, k,l}(\omega_{k,l}) & =\begin{aligned}[t]
        \sum_{s^1,s^3} \exp{(\frac{\beta \omega_{k,l}}{2})} \int &
        \frac{m_e{(4\pi)}^2 k_1^4 dk_1}{{(2\pi)}^6\sqrt{k_1^2 - 2m_e\omega}}
        V_{i,j} V_{k,l} [ N_1 (1 - N_1)]
    \end{aligned} \\
                                    & =\begin{aligned}[t]
        \sum_{s^1,s^3} \exp{(\frac{\beta \omega_{k,l}}{2})} \int &
        \frac{m_e k_1^3 dk_1}{4\pi^4}
        V_{i,j} V_{k,l} [ N_1 (1 - N_1)]
    \end{aligned}
\end{align}
we expand \(N_1 (1 - N_1)\) about \(k_1 = k_f\)
\begin{align}
    N_1 (1 - N_1) & = \frac{1}{1 + \exp{\beta \Delta E}}
    \frac{1}{1 + \exp{-\beta \Delta E}}                                                \\
                  & \sim \frac{1}{2 + \beta \Delta E + \frac{{(\beta \Delta E)}^2}{2}}
    \frac{1}{2 - \beta \Delta E + \frac{{(\beta \Delta E)}^2}{2}}                      \\
                  & = \frac{1}{4}
    \frac{1}{1 + \frac{\beta \Delta E}{2} + \frac{{(\beta \Delta E)}^2}{4}}
    \frac{1}{1 - \frac{\beta \Delta E}{2} + \frac{{(\beta \Delta E)}^2}{4}}            \\
                  & \sim \begin{aligned}[t]
        \frac{1}{4}
         & (1 - \frac{\beta \Delta E}{2} - \frac{{(\beta \Delta E)}^2}{4} + {(\frac{\beta \Delta E}{2})}^2) \\
         & (1 + \frac{\beta \Delta E}{2} - \frac{{(\beta \Delta E)}^2}{4} + {(\frac{\beta \Delta E}{2})}^2)
    \end{aligned}                                    \\
                  & = \frac{1}{4}(1 - \frac{{(\beta \Delta E)}^2}{4})                  \\
                  & \sim \frac{1}{4}\exp{(- \frac{{(\beta \Delta E)}^2}{4})}
\end{align}
The sharp exponential decay justifies
this expansion. Going back to the
integral we have
\begin{align}
    \Gamma_{i,j, k,l}(\omega_{k,l}) & =\begin{aligned}[t]
        \sum_{s^1,s^3} \exp{(\frac{\beta \omega_{k,l}}{2})} \int &
        \frac{m_e k_1^3 dk_1}{{(2\pi)}^4}
        V_{i,j} V_{k,l} \exp{(- \frac{{(\beta \Delta E)}^2}{4})}
    \end{aligned} \\
                                    & =\begin{aligned}[t]
        \sum_{s^1,s^3} \exp{(\frac{\beta \omega_{k,l}}{2})} \int &
        \frac{m_e k_1^3 dk_1}{{(2\pi)}^4}
        V_{i,j} V_{k,l} \exp{(- \frac{{(\frac{\beta \hbar^2}{2 m_e}(k_1^2 - k_f^2))}^2}{4})}
    \end{aligned} \\
                                    & =\begin{aligned}[t]
        \sum_{s^1,s^3} \exp{(\frac{\beta \omega_{k,l}}{2})} \int &
        \frac{m_e k_f^2 du}{2{(2\pi)}^4}
        V_{i,j} V_{k,l} \exp{(- \frac{\beta^2 \hbar^4}{{(4m_e)}^2} u^2)}
    \end{aligned} \\
                                    & =\begin{aligned}[t]
        \sum_{s^1,s^3} \exp{(\frac{\beta \omega_{k,l}}{2})} \frac{m_e k_f^2 }{{(2\pi)}^4}
        V_{i,j} V_{k,l} \sqrt{\pi} \frac{2m_e}{\beta \hbar^2}
    \end{aligned}
\end{align}
checking the units of \(\Gamma \) we find
\begin{align}
    [\Gamma] & = {[kg]}^2[m^{-2}]{[kgm^5s^{-2}]}^2{[kgm^2s^{-2}]}^{1}{[kg m^2 s^{-1}]}^{-2} \\
             & = [{kg}^2 m^{-2} m^{6} s^{-2} {kg}^{1} m^{2} s^{-2}]                         \\
             & = [{kg}^3 m^{6}s^{-4}]
\end{align}
to recover the correct units of \(s^{-1}\) we
therefore divide by \(\hbar^3 \) which was previously
taken to be 1. We can also write
\(V_{i,j}
= \mathcal{C}_{i,j} \frac{2e^2}{\epsilon_0 \alpha^2}
= \mathcal{C}_{i,j} \frac{8 \pi^2 \epsilon_0 \hbar^4}{e^2 m_e^2}\)
where \(\mathcal{C}_{i,j}\) is the hydrogen
overlap factor given in \cref{sec:simplified interaction}
\begin{align}
    \Gamma_{i,j, k,l}(\omega_{k,l}) & =
    \sum_{s^1,s^3} \exp{(\frac{\beta \omega_{k,l}}{2})}
    \frac{m k_f^2 }{{(2\pi)}^4}
    \mathcal{C}_{i,j} \mathcal{C}_{k,l}
    \sqrt{\pi} \frac{2m}{\beta \hbar^5} {(\frac{8 \pi^2 \epsilon_0 \hbar^4}{e^2 m_e^2})}^2 \\
                                    & =
    \sum_{s^1,s^3} \exp{(\frac{\beta \omega_{k,l}}{2})}
    \mathcal{C}_{i,j} \mathcal{C}_{k,l}
    \sqrt{\pi} \frac{8 k_f^2 \epsilon_0^2 \hbar^3}{\beta e^4 m_e^2}
\end{align}

If we go back to calculate the
\(\omega = 0\) contribution
we find (by taking the low
temperature limit and noting
particle number is constant
with temperature)
\begin{align}
    \Gamma_{i,j, k,l}(\omega) & =
    \sum_{s^1,s^3} \int_0^{k_f}
    \frac{d^3\vec{k}_1}{{(2\pi)}^3}
    \frac{d^3\vec{k}_3}{{(2\pi)}^3}
    V_{i,j} V_{k,l} [
    N_1 N_3 \delta_{w, 0} \frac{m_e}{k_3}] \\
                              & =
    \sum_{s^1,s^3} \int_0^{k_f}
    \frac{4\pi k_1^2 dk_1}{{(2\pi)}^3}
    \frac{4 \pi m_e k_3 dk_3}{{(2\pi)}^3}
    V_{i,j} V_{k,l} \delta_{w, 0}          \\
                              & =
    \sum_{s^1,s^3} \frac{4\pi k_f^3 m_e}{3{(2\pi)}^3}
    \frac{2 \pi k_f^2}{{(2\pi)}^3}
    V_{i,j} V_{k,l} \delta_{w, 0}          \\
                              & =
    \sum_{s^1,s^3} \frac{4\pi k_f^3 m_e}{3{(2\pi)}^3}
    \frac{2 \pi k_f^2}{{(2\pi)}^3}
    V_{i,j} V_{k,l} \delta_{w, 0}          \\
\end{align}
If we add back in the factor of  \(\hbar^3\)
\begin{align}
    \Gamma_{i,j, k,l}(0) & =
    \sum_{s^1,s^3} \frac{4\pi k_f^3 m_e}{3\hbar^3{(2\pi)}^3}
    \frac{2 \pi k_f^2}{{(2\pi)}^3}
    \mathcal{C}_{i,j} \mathcal{C}_{k,l}
    \delta_{w, 0} {(\frac{8 \pi^2 \epsilon_0 \hbar^4}{e^2 m_e^2})}^2 \\
                         & =
    \sum_{s^1,s^3}
    \frac{8k_f^5 \epsilon_0^2 \hbar^5}{3e^4 m_e^3}
    \mathcal{C}_{i,j} \mathcal{C}_{k,l} \delta_{w, 0}
\end{align}
If we use dimensional analysis on this term
we find it is proportional to \(m^{-1}s^{-1}\).
This is because we have ignored an important
subtlety when imposing the \(\omega \) delta
function. If we had done it properly (before
setting \(k_3 = k_4\)) we would get
out an extra factor of \(L\). This term
therefore diverges as \(L \rightarrow \infty \).
\begin{align}
    \Gamma_{i,j, k,l}(0) & =
    L \sum_{s^1,s^3}
    \frac{8k_f^5 \epsilon_0^2 \hbar^5}{3e^4 m_e^3}
    \mathcal{C}_{i,j} \mathcal{C}_{k,l} \delta_{w, 0}
    \label{eqn:divergent expression for first integral}
\end{align}


\end{appendix}

\end{document}
